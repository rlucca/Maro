\section{Introduction}

This work presents a tool developed to model affective behaviour of virtual
actors. The work revised agent oriented programming, computing affective
and how \jason platform \citep{bordini-jason} works which was done because the
tool is a framework built on top of that platform.

According to \citet{gruber1993translation}, ``ontology is a explicit
specification of a conceptualization''. In the present, it has been understood as
a shareable common knowledge which can be used in communication between people
or machines \citep{wks2008towards}. So, ontology came as an excellent way of do
two things. First, it is a good manner to expose the rules of the system in a
general way. Second, it is easy to a human do the configuration needed.

The architecture of the system was divided in one environment and a belief
base. The environment developed used two ontologies to allow the agents
receive perceptions its preferences and schedules for one day of activity.
The first ontology was done to make possible the agents have its preference
based on annotations that exist in all objects and characters in the world.
The ontology of schedule was proposed by \citet{paiva2005ontology}, and it
was used to do the world virtual used in one of the validation cases and the
schedule of each agent. It allows the agent to have fixed routes (the agent
always do this route) and random routes (the agent eventually do this).

A belief base is similar to a database excepts that it contains all believes
and plans of actions that one agent have. The second part of the architecture
developed divided the belief base in two levels. The first level is the
ontology based on the emotion model of \citet{ortony1988cse} developed. It
was built to have all the emotions from the original model. Also, it was
constructed to be used with the ontology of preferences and the agent can
expose its feeling through annotation of its name.

Although the multi-agent platform has a way to handle ontologies
\citep{moreira2006agent}, it was decided to not use this implementation. So,
it was needed to implement a light version of it. The new version is less
flexible because it does not allow agents to create new concepts and new data
or object properties during run-time execution. However, this decision allowed
the agent beliefs by mapped to and from ontology dynamically.

To test what was developed, this work built three different examples. The
first demo built was thought as a unit test to protect from possibly changes during
its development. So it tests all processes needed during the integration as insertion,
recover, listing or remove beliefs. Next, it was a soccer fight simulation
which two agents cheers by their teams. The emotion ontology was used in a
manual way. The last one used all ontologies explained to built a
visualization of an apartment and defines routines and emotional responses to
annotations in objects and characters.

A mechanism to represent emotion in agents based on a very acceptable emotion
model was developed. Also, a library to understand the ontology and know what
needs by send to or receive from the same was done. The ontologies were
implemented as modules and each one can be used independently.
The source of all examples, the articles and other artefacts developed is
available online on \emph{GitHub}\footnote{http://github.com/rlucca/Maro}.
