\section{Trabalhos Relacionados}
%\section{Ontologias de Humanos Virtuais ou Similares} % TI 2
%eu nao me lembro o que eh isso comentado, talvez nao seja uma
%secao seja so uma nota para ir la ver e entao botar aqui

\todo{eu acho que isso aqui ta muito relacionado com Agents, cade a ontologia? ou nao precisa?}

O comportamento emotivo do personagem tem um papel importante para se ter a
ilusão de vida conforme afirmou \citet{bates1994role}. Esse trabalho utiliza o
modelo OCC para melhorar a credibilidade dos agentes e cada uma das emoções
pode ser ligada a um comportamento. No exemplo dado anteriormente, um agente
pode ligar o medo a um comportamento agressivo enquanto outro liga essa mesma
emoção de medo a um comportamento de estado alarmado.

\citet{GraCli98} criaram um mecanismo evolucionário utilizando redes neurais
com uma base química para guiar o comportamento do ator. Os atores simulados
podem envelhecer, aprender e, inclusive, se reproduzir (aqui são utilizados
algoritmos genéticos).  Por exemplo, o personagem pode aprender algumas
palavras básicas e demostrar que esta envelhecendo por meio da mudança da cor
de seus cabelos dando uma certa ilusão de vida.

Conforme já dito, as emoções podem melhorar a credibilidade dos seres
virtuais.  \citet{zhang2009emotional} desenvolveram uma aplicação com a
finalidade de demostrar esse conceito. O planejamento das ações a serem
executadas pelo personagem é afetado pelos valores das emoções sendo
experimentadas.

Todos os trabalhos apresentados até aqui, utilizaram as emoções para aumentar
a representação ou expressividade de um ator. Entretanto,
\citet{neto2010construction} tentam estudar o impacto da emoção na decisão dos
agentes.  Assim, para isso ser possível, foi necessário alterar a forma de
planejamento das decisões, além de como recuperar os dados da sua memória.  A
modificação no acesso a memória permite o ``esquecimento'' de determinadas
crenças quando o estado emocional for diferente daquele guardado junto da
memória a ser recuperada. Essa característica torna o planejamento do agente e
as atitudes dos personagens mais realísticas.

\citet{benta2007ontology} e \citet{wks2008towards} descrevem modelos afetivos
através de ontologias. No primeiro trabalho, uma ontologia foi criada
descrevendo emoções primárias e secundárias. O primeiro tipo de emoção exige
menos processamento cognitivo que o segundo. Além disso, o modelo descreve ao
todo 11 emoções sendo 7 dessas consideradas primárias.

