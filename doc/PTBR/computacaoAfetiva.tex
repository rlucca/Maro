\section{Computação Afetiva}

\citet{Pic98} definiu Computação Afetiva como uma ``computação relacionada,
surgida ou que influência as emoções''. Além disso, computadores com emoções
permitem aos mesmos um determinado nível de comportamento inteligente e
criatividade que seria impossível sem as emoções e esse é o principal desafio
dessa área. Logo, o seu entendimento pode explicar fenômenos como, por
exemplo, atenção, memória e outros.

Essa área é normalmente dividida em duas sub-áreas. A primeira estuda o
reconhecimento e a expressão de emoções dentro da IHC;
a segunda, foca na síntese de emoções para aprimorar os seres robóticos e/ou
para estudar o comportamento humano por meio de simulações. Há muita
aplicabilidade dessas técnicas, por exemplo: a área que reconhece as emoções
pode ser utilizada para adaptar o sistema ao estado da pessoa permitindo ao
mesmo instruí-la, questioná-la, encorajá-la ou ocultar informações irrelevantes.

O objetivo de \citet{bick2003relational} com o projeto \emph{Relational
Agents} é possibilitar aos usuários a criação de um relacionamento social e
emocional de longa duração.  Em \citet{bickmore2009virtual}, a confiança no
agente torna possível discutir tarefas mais importantes como melhoria da saúde
ou até a compra de uma casa. Outro trabalho na área de IHC é o reconhecimento
de emoções para aumentar a imersão em jogos, por exemplo permitindo ao próprio
jogo adaptar eventos ou trechos tornando-o mais divertido.

\begin{figure}
  \begin{center}
    \includegraphics[width=75mm]{figuras/tigger-mit.png}
  \end{center}
  \caption[Brinquedo que responde as emoções das crianças.]{Brinquedo que responde as emoções das crianças \cite{kirsch1999affective}.}
  \label{fig:tigger-mit}
\end{figure}

O projeto \emph{The Affective Tigger: a reactive expressive toy} de
\citet{kirsch1999affective} é um brinquedo capaz de reconhecer e reagir às
emoções exibidas pelas crianças. Por exemplo, quando a criança encontra-se
feliz, o boneco expressa felicidade (ver Figura~\ref{fig:tigger-mit}). Ao todo
existem 5 estados emocionais: muito feliz, feliz, neutro, triste e muito
triste. Todos, com exceção do neutro, possuem alguma síntese vocal como um
rosnado (tristeza) ou uma risada (muito feliz). Assim, esse brinquedo, por ser
considerado um ser robótico que reage à criança com seus próprios estados
emocionais, fica enquadrado na segunda área.  Portanto, o desenvolvimento
desse brinquedo serviu para aprimorar os seres robóticos.

O projeto AIDA\footnote{Mais detalhes, ver http://senseable.mit.edu/aida} (do
inglês \emph{Affective Intelligent Driving Agent}) pode ser entendido como
enquadrado na área de IHC, pois o interesse é entender o estado afetivo da
pessoa dirigindo. Além disso, interessa-se em ter um relacionamento com o
usuário sugerindo alterações nas rotas baseando-se na rotina aprendida depois de
um tempo de aprendizado.  A pesquisa relatada em \citet{dias-agents} visou
melhorar a simulação de agentes através do uso da emoção guiando o processo
deliberativo e melhorando o entendimento e gerência das emoções.  O presente
trabalho se enquadra na área de síntese de emoção, pois o interesse é em
entender o estado emocional e como ele pode afetar o comportamento de um
personagem.

\subsection{Modelo Cognitivo Emocional} \label{ch-ca-mce}

Estudos neurológicos recentes \cite{ledoux1998emotional,damasio2004erro}
mostram a importância das emoções na tomada de decisão.
\citet{damasio2004erro} definiu emoção como sendo um estado físico do corpo
que se altera de forma contínua. Sendo assim, o sentimento foi definido como a
percepção dessas alterações.

Na psicologia há diferentes modelos que tentam explicar a afetividade.
\citet{scherer2000tnoe} categorizou esses modelos afetivos em quatro
categorias principais: modelos dimensionais, modelos discretos, modelos
baseados em significados e modelos baseados em componentes. A primeira
categoria visa descobrir variáveis que representam eixos das classes emotivas
e estabelecem meios de se mover por esses eixos. A seguinte, especifica um
conjunto básico de emoções e regras de evolução para esse conjunto.
Já, a categoria dos baseados em significados se preocupa com as situações
em que o sentimento foi ocasionado e tenta descrever as estruturas semânticas dos
mesmos. A última, entende que os sentimentos são aprendidos ao longo do tempo.
Sendo assim, os modelos baseados em componentes estudam o elo entre os
sentimentos e as suas situações. Esse elo é montado de diferentes formas e
varia de pessoa para pessoa.

Um modelo baseado em significados, bastante conhecido na Inteligência
Artificial, foi definido por \citet{ortony1988cse}. Nesse modelo\footnote{A
partir daqui o modelo será referenciado por modelo \emph{OCC}.} são descritos
22 emoções com suas situações. Essas emoções são divididas em formas de se
perceber o mundo a sua volta por consequências (importância das metas),
ações (responsabilidade) e objetos (atração ou repulsa). Assim sendo,
essas maneiras refletem diferentes jeitos de se
analisar as situações que podem ser relativas aos objetivos, valores morais ou
gostos da pessoa.

\begin{figure}[t]
  \centering
    \includegraphics[width=128mm]{figuras/occ.png}
  \caption[Estrutura de emoções.]{Estrutura de emoções \cite{ortony1988cse}.}
  \label{fig:occ_model}
\end{figure}

A Figura~\ref{fig:occ_model} resume esse modelo e mostra as
percepções possíveis de um indivíduo.  Partindo da direita para esquerda, o
ramo mais básico, \emph{Aspects Of Objects}, é ativado quando se avalia o
gosto de alguém para algum objeto (inanimado ou não). Por exemplo, Millie
gosta de rosas azuis. No seguinte, \emph{Actions Of Agents}, o julgamento das
ações exercidas por outro indivíduo ou por si mesmo é baseado nos
valores morais da pessoa que está julgando. Exemplo: reprovar a atitude dos
bancários que fazem greve a cada ano. Cabe salientar, que ao julgar ações, o
modelo permite um grau de ``empatia'' chamado pelos autores de ``unidade de
força cognitiva''\footnote{Traduzido literalmente de \emph{Strength of
cognitive unit}.}. Dessa forma, é possível, por exemplo, ficar com orgulho
porque uma atleta ganhou uma medalha ou ficar envergonhado ao descobrir que o
vizinho bate no(s) filho(s).

O último ramo da árvore, mais a esquerda na Figura~\ref{fig:occ_model}, é o
\emph{Consequences Of Events} que representa as coisas que aconteceram (e
foram consideradas importantes), acontecem ou podem acontecer (objetivos
almejados)\dev{}. Essas emoções são avaliadas segundo as suas consequências
para o alcance ou impedimento dos seus objetivos. Exemplos possíveis desse
ramo do modelo são a emoção sentida ao receber uma boa nota em um teste, ao
ser assaltado ou ao perceber seu voo ser cancelado por algum problema não
esperado.

Todas as emoções do modelo trabalham com duas intensidades. A intensidade da
emoção que representa o físico e a intensidade do sentimento que representa o
quanto o agente esta percebendo daquela emoção. Dessa forma, um indivíduo só
possui sentimento quando a intensidade da emoção ultrapassa um
determinado limite\dev{}.  Essa intensidade é obtida por uma função matemática
que utiliza variáveis de dois tipos: locais, que influenciam as emoções do ramo
específico; e globais, que influênciam todas as emoções do modelo.  Um exemplo
de variável local é o desejo, enquanto um exemplo de variável global pode ser
o senso de realidade de uma pessoa.

\citet{bates1994role} foi um dos primeiros a trabalhar na utilização de emoções
na área de animação. Nessa área, o estudo do comportamento humano é realizado
visando imitar as ações humanas. Assim, a principal afirmativa do trabalho era
que o comportamento emotivo de um personagem é um papel importante para que o
mesmo pareça ter vida própria. Dessa forma, esse trabalho utilizou o modelo
descrito visando melhorar a credibilidade de seus atores. Por exemplo, um dos
agentes lida com o medo sendo agressivo com os outros enquanto outro agente
lida com a mesma emoção sendo retraído.

%% Comentado mas iria em trabalhos relacionados...
%\citet{GraCli98} criaram um mecanismo evolucionário utilizando redes neurais
%com uma base química para guiar o comportamento do ator. Os atores simulados
%podem envelhecer, aprender e, inclusive, se reproduzir (aqui são utilizados
%algoritmos genéticos).  Por exemplo, o personagem pode aprender algumas
%palavras básicas e demostrar que esta envelhecendo por meio da mudança da cor
%de seus cabelos dando uma certa ilusão de vida.

Visando entender melhor o impacto da emoção na tomada de decisão,
\citet{zhang2009emotional} desenvolveram uma aplicação que os sentimentos
afetam o planejamento das ações à serem realizadas.
\citet{neto2010construction} focaram no mesmo objetivo, porém visando estudar
o impacto da memória no planejamento. Sendo assim, um meio para o agente
``esquecer'' determinadas crenças quando o estado emocional for diferente
daquele guardado anteriormente foi realizado. Eles acreditam que essa
característica torna o planejamento e as atitudes dos personagens virtuais
mais realista.

Um dos trabalhos mais conhecidos baseado no modelo \occ é, sem dúvida, o de
\citet{kshirsagar2002multilayer}. Ele utilizou
as emoções levantadas no modelo em conjunto com um modelo de personalidade
baseado na psicologia que leva em consideração 5 fatores: extroversão,
agradabilidade, conscientização, neurose e receptividade. O primeiro, descreve
a preferência para o comportamento em situações sociais. O seguinte, a
interação com os outros indivíduos. A conscientização é a organização e
persistência das metas. A tendência de pensamentos negativos é a neurose ou
fator neurótico. Por fim, o último, descreve se a pessoa tem interesse em
cultura ou é ``cabeça aberta'' para novas ideias.

\subsection{Ontologias Afetivas}

De acordo com \citet{Gutierrez:2007:OVH:1229160.1229164}, uma
ontologia possui inúmeras utilidades tanto em pesquisa quanto na
indústria. Na pesquisa o foco é uma melhor descrição do domínio propriamente
dito, enquanto na indústria o foco é a melhor utilização dos recursos de seus
colaboradores. As ontologias emocionais visam descrever emoções ou aspectos
afetivos de um indivíduo se baseando ou não em estudos da psicologia.

Em \citet{benta2007ontology} foi feita a construção de uma ontologia escrita em
\emph{OWL}. Nesse trabalho as emoções são divididas em primárias e secundárias, as
secundárias se originam a partir das primárias. As emoções primárias,
não cognitivas, são: \emph{Angry}, \emph{Disgust}, \emph{Fear},
\emph{Happy}, \emph{Neutral}, \emph{Sad} e \emph{Surprise}. As emoções
secundárias, cognitivas, descritas são ao todo 4. O interessante aqui é que
essas 4 emoções são inferidas a partir das anteriores. Além disso,
há o conceito de emoção ativa que é a emoção predominante naquele momento. O
valor da emoção é calculado da seguinte forma, a sensibilidade (predisposição
a emoção varia de 0 à 1) multiplicado pela intensidade da emoção. A emoção
predominante é o maior valor entre as emoções.

No modelo \occ a distinção entre os tipos de
emoções não existe porque se pressupõe que toda emoção exige um certo nível de
cognição. Não existem, também, um limite no que pode ou não ser percebido em
quantidade de emoções. Mas, existe um limite de perseguir uma meta por vez.
Fora isso, se pode pensar que emoções opostas compartilham os mesmos atributos
e, por isso, essas emoções não serão sentidas ao mesmo tempo.

Não tendo nenhuma informação de uma teoria de emoções específicas modeladas em
sua ontologia, \citet{wks2008towards} criaram uma ontologia de alto nível se
aproveitando de outra ontologia de alto nível e de uma de analise léxica.
Assim, o principal conceito desse trabalho é o de sensor que é um objeto
físico no ambiente e que recebe as informações do meio e as ``transportam''
para o mundo mental do agente. Sendo assim, é possível reconhecer a
percepções similares e descrever novas situações. Todavia, o presente trabalho
não tem objetivo de criar uma ontologia de alto nível\dev{}.

\citet{springerlink:10.1007/978-3-642-01639-448} desenvolveram um motor de
emoções que utiliza um modelo de mistura de emoções em conjunto com o modelo
\emph{OCC}. O trabalho deles utilizou o conceito de camadas onde cada camada tem uma
responsabilidade distinta e que visa complementar a anterior. Essas são ao todo
quatro: classificação, interação, mapeamento e expressão. A primeira visa
determinar que categoria ou ramo será afetado. A seguinte, determina a
intensidade da emoção. A de interação analisa os efeitos nas categorias
emocionais do personagem. A próxima, mapeia as 22 emoções do modelo para pelo
menos uma expressão. A expressão propriamente dita é feita pela última camada.

Esse trabalho ainda utilizou um modelo dimensional para misturar as emoções
primárias. Assim, as emoções secundárias podem ser descobertas a partir do
nível dos eixos afetados. O trabalho mostra quase todas as emoções do
ramo de consequência de eventos, com exceção das emoções de \emph{Hope} e
\emph{Fear}. As emoções primárias são as emoções do modelo \occ e como
secundárias estão as emoções construídas a partir da mistura dessas emoções.
Essa diferenciação, como dito anteriormente, não existe no modelo \emph{OCC}.

Um modelo genérico que representa o ambiente e eventos que estão envolta de um
personagem, sua personalidade e suas preferências foi feito por
\citet{lera2009semantic}. Nesse trabalho o foco eram emoções que podem ser
representadas no rosto. Assim, a identificação do contexto é feita através de
eventos e do retorno afetivo. Fora isso, as expressões faciais são modificadas
dependendo dos eventos do ambiente e, também, da personalidade, metas e
preferências dos atores virtuais.

\citet{adam2009alfototoe} formalizaram o modelo \occ de maneira lógica.
Esse trabalho descreve a probabilidade de maneira comparativa, isto é, o
evento A é mais provável que o evento B. Além disso, ha uma ordem temporal nas
ações sendo desempenhadas. O autor propõem uma diferenciação entre ação e
evento. O primeiro é causado intencionalmente pelo agente\dev{},
enquanto o segundo o agente não tem controle da ação. Por exemplo, ir para
aula é uma ação que pode ser julgada pela responsabilidade e espirrar é um
evento que só pode ser julgado pela sua consequência.

\subsection{Arquiteturas Emocionais} % sao so as similares (usam o OCC)

\citet{elliott1992tar} em sua tese ``O raciocinador afetivo'' foi um dos
primeiros a trabalhar com emoções na área de sistemas multi-agentes. Em seu
modelo 24 tipos de emoções são usadas, isto é, duas à mais que o modelo
\emph{OCC} por causa que ele dividiu as emoções do ramo de objetos para
diferenciar amor de gostar e ódio de não gostar. Nesse trabalho, um mesmo
evento pode gerar mais de uma emoção ao mesmo tempo. Por exemplo, estar feliz
por que recebe um bom salário e triste porque o trabalho é chato. Além disso,
os agentes podem observar os outros e explicar situações usando regras
emotivas e um sistema de classificação baseado por casos.

Já, \citet{gratch2000empitae} focou a sua arquitetura emocional diretamente
nos planos e metas dos agentes. O modelo criado por ele permite agentes
avaliarem o significado emocional de eventos que se relacionam com os planos e
metas, modelando e predizendo os estados afetivos de outros agentes e
alterando o comportamento de acordo. Para isso um sistema baseado em casos
também foi utilizado para determinar a existência de relações e se existirem
elas são reconhecidas por determinadas características. O trabalho estava
restrito a eventos comunicados ou ações dos agentes e utilizou apenas cinco
emoções: esperança, alegria, medo, sofrimento e raiva.

\citet{gratch2004domain} utilizaram os dois primeiros trabalhos explicados para
desenvolver este. Com um acrescimento, ele explicou um processo de avaliação
que observa o ambiente e configura ``variáveis de avaliação'' e, após, o
processo de imitação acontece. O processo de imitação pode ser de duas formas,
ser resolvido por problema ou por emoção. Se for feita uma solução por
problema o autor apontou várias estrategias possíveis: cópia ativa (tentar
fazer), buscar suporte (procurar conselheiro, informação, etc) e planejar. Se
for adotada uma solução emocional tem se ao todo 11 estrategias muito bem
explicadas em seu trabalho. Além disso, o trabalho menciona que o agente
mantem a historia do que aconteceu, esta acontecendo e pode vir a
acontecer para ser consultada durante seu planejamento. O trabalho usa como
exemplo um medico que tem que decidir aplicar um medicamento agora e fazer o
paciente morrer mais tarde ou deixa-lo morrer agora.

O trabalho \emph{FearNot} foi desenvolvido visando reduzir os bullying nas
escolas \cite{dias2005feeling}. As crianças são expostas as cenas de bullying
e fazem o papel do amigo imaginário da vitima que pode sugerir formas de ação
para as situações. A arquitetura criada foi pensada tendo em
mente: ter capacidades reativas e cognitivas, gerar credibilidade e empatia,
possuir interação com o usuário (note que os agentes não seguem cegamente o que
o usuário diz porque isso baixaria a credibilidade) e ser independente de
domínio.
%
O processo de deliberação do agente é focado na avaliação feita a partir do
que se percebe do ambiente em níveis reativos e deliberativos (emoções com
probabilidade). Depois, o estado emocional é gerado e a memória atualizada
paralelamente. Esse trabalho, também, utiliza um processo de imitação que pode
ser no nível reativo (tendencias de ações) ou deliberativo (foco no problema
ou emoção). Dessa forma, de maneira similar com o trabalho anterior quando o
processo de imitação é deliberativo uma reavaliação do problema é feita que
pode afetar o estado emocional ou o entendimento do problema.

