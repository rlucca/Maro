\subsection{Paradigma de Programação} \label{sec:aoppp}

\citet{shoham1993agent} propôs em seu trabalho a linguagem \emph{Agent-0}, uma
das primeiras a serem baseadas em atos de fala e uma nova visão para se
programar.  Esses atos de fala podem ser vistos como comandos falados entre
atores com as mais diversas finalidades (informar, perguntar, requerer,
aceitar, etc).  Dessa forma, \citeauthoronline{shoham1993agent} tentou promover a
ideia de computação com uma interação mais social entre os membros de um sistema.
Em outras palavras, esse novo paradigma de programação precisa que exista
cooperação e competição entre os agentes para a realização das tarefas
desejadas.

Assim, o agente encontra-se situado em um ambiente no qual pode receber de ou
enviar para outros agentes informações. As ações são
estabelecidas utilizando regras que descrevem comportamentos. Logo, 
a própria entidade decide qual ação deve ser tomada. Essas regras
são construídas em fórmulas lógicas descrevendo-se o contexto que torna um
curso de ação válido e sua consequência. Essa consequência pode ser uma ação
ou um conjunto sequencial de ações que precisam ser feitos.

Atualmente, segundo \citet{bordini2009multi}, há diversas linguagens para a
AOP. Por exemplo: \emph{2APL}, \emph{Agent-0}, \emph{AgentSpeak(L)},
\emph{GOAL}, \emph{MetateM} e etc. Elas são baseadas em diversos
formalismos diferentes. \emph{MetateM}, por exemplo, é baseada em lógica temporal,
permitindo que formulas temporais definam o comportamento do agente.
\emph{GOAL} tem a visão que o conhecimento pode ser estático (imutável) ou
dinâmico e escolhem suas ações a partir das suas crenças e metas de uma
maneira similar ao que acontece no \emph{Jason}.
\emph{Agent-0}, como dito anteriormente, é a primeira linguagem da
área de AOP. \emph{2APL} é sucessora da \emph{3APL} e dividiu a linguagem em
dois formatos básicos \cite{dastani20082apl}. O primeiro define como a
plataforma irá operar, isto é, quantos e em qual ambiente o agente se
encontra. O segundo formato define a programação propriamente dita do indivíduo.

No presente trabalho, a plataforma \jason \cite{bordini-jason} será utilizada.
Ela usa uma extensão da linguagem \emph{AgentSpeak(L)}, definida por
\citet{rao1996agentspeak}, para especificar as atitudes que o agente deve
tomar frente as constantes modificações do ambiente. Assim, essa linguagem é,
normalmente, vista como reativa às crenças, metas ou percepções por causa que
trabalha com a ideia de eventos sobre esses conceitos.

