\subsection{Paradigma de Programação}

\citet{shoham1993agent} propôs em seu trabalho a linguagem \emph{Agent-0} uma
das primeiras a serem baseadas em atos de fala e uma nova visão para se
programar.  Esses atos de fala podem ser vistos como comandos falados entre
atores com as mais diversas finalidades (informar, perguntar, requerer,
aceitar, etc).  Dessa forma, \citeauthor{shoham1993agent} tentou promover a
idéia de computação com uma interação mais social entre membros de um sistema.
Em outras palavras, esse novo paradigma de programação precisa que exista
cooperação e competição entre os agentes para a realização das tarefas
desejadas.

Assim, o agente encontra-se situado em um ambiente no qual pode receber
informações de outros agentes, assim como enviar para outros.  As ações são
estabelecidas utilizando regras que descrevem comportamentos. Assim, é
possível, a própria entidade decidir qual ação deve ser tomada. Essas regras
são construídas em fórmulas lógicas descrevendo-se o contexto que torna um
curso de ação válida e sua consequência que é o conjunto de ações a serem
desempenhadas.

Atualmente, há diversas linguagens para a programação de agentes, por exemplo:
\emph{2APL}, \emph{Agent-0}, \emph{AgentSpeak(L)}, \emph{GOAL} e
\emph{MetateM}~\cite{bordini2009multi}. Elas são baseadas em diversos
formalismos diferentes. MetateM por exemplo é baseada em lógica temporal,
permitindo que formulas temporais definindo o comportamento do agente sejam
diretamente executadas.

A plataforma \jason \cite{bordini-jason} utilizada no trabalho pode ser
pensada como um sistema de planejamento reativo, em que planos são executados
a partir de alterações nas crenças e objetivos do agente. A plataforma
baseia-se na arquitetura BDI e utiliza uma extensão da linguagem de
programação abstrata, definida por \citet{rao1996agentspeak}, chamada
AgentSpeak(L), para especificar o reciocínio sobre ações dos agentes. A grande
vantagem dessa plataforma é a rapidez com que melhorias têm sido feitas e a
facilidade de customização de diversos componentes da plataforma.

