\chapter{Introdução}

Segundo \citet{terzopoulos1998behavioral}, na indústria cinematográfica têm-se
excelentes exemplos do estudo do comportamento humano.  A área de animação
estuda o comportamento visando imitar as ações humanas.  Entretanto, ainda há
muito trabalho a ser feito na construção de um ser que pareça realmente vivo.
%
\citet{bates1994role} foi um dos primeiros a trabalhar na melhoria da
interpretação de atores virtuais utilizando emoções.
\citeauthor{bates1994role} afirmou que as atitudes de um personagem com ilusão
de vida própria é um grande desafio.  Por exemplo, João pode ser agressivo
quando estiver com medo enquanto André ao perceber o medo fica retraído.
%Além disso, a emoção pode ser utilizada para guiar o comportamento de
%determinado personagem.

Segundo \citet{damasio2004erro}, emoções podem ser primárias (não-cognitivas)
ou secundárias (cognitivas).  As emoções primárias surgem a partir de reações
a determinados estímulos característicos e são geradas rapidamente.  Já as
emoções secundárias são aprendidas ao longo da nossa vida, isto é, são geradas
por uma avaliação de uma situação de acordo com nossos objetivos e valores
morais.  Por exemplo, estou com pena de Alberto por possuir determinada
doença.

A área da computação que estuda esse assunto foi denominada Computação Afetiva
por \citet{Pic98}.  Há dois ramos principais de trabalho nessa área.  O
primeiro estuda o reconhecimento e a expressão de emoções dentro da Interação
Homem-Computador (IHC).  O segundo foca na síntese de emoções para estudar o
comportamento humano por meio de simulações ou aprimorar os seres robóticos ou
atores virtuais.

%% se eu mencionar isso preciso ter um levantamento % bibliografico a respeito
%disso...  Entre esses dois eixos, a simulação de multidão serve de elo porque
%se esta interessado em simular milhares de pessoas de forma mais realista
%possível \cite{thalmann2007crowd}.  Dessa forma, o presente trabalho visa
%desenvolver uma ferramenta para permitir que os personagens atuem de forma
%convincente com o mínimo de interação de um animador/programador no seu
%controle.  Para isso, relacionar as áreas mencionadas visando recriar a vida
%é uma necessidade.

O presente trabalho visa desenvolver uma ferramenta que utilize ontologias
para definir o comportamento emocional dos personagens virtuais.  Um objetivo
secundário é a redução de configurações necessárias para o seu uso porque isso
tornaria difícil ou cansativo para um animador/programador utilizar.
%
Este artigo está organizado da seguinte maneira. \todo{rever} Na
seção~\ref{EA} abordam-se os conceitos de Agente e Personagem, o paradigma de
programação de agentes, juntamente com uma breve explicação da plataforma que
esta sendo utilizada, bem como uma introdução à computação afetiva e a
ontologias; essa seção também discute trabalhos relacionados.  A idéia da
ferramenta é explicada em mais detalhes na parte~\ref{MET}. Na \ref{REES}, os
resultados esperados e considerações finais são expostos.
