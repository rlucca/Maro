\chapter{Introdução}

Segundo \citet{terzopoulos1998behavioral}, a indústria cinematográfica possui
excelentes exemplos do estudo do comportamento humano na área de animação.
Essa área visa imitar as ações humanas computacionalmente. Entretanto, ainda
há muito trabalho a ser feito na construção de um ator virtual que pareça
realmente vivo. Um dos primeiros trabalhos envolvendo emoções, em personagens
virtuais, é o de \citet{bates1994role} visando melhorar a interpretação. Por
exemplo, João pode ser agressivo quando estiver com medo e André fica retraído
ao sentir medo.

As emoções podem ser categorizadas, segundo \citet{damasio2004erro}, em
primárias (não-cognitivas) e secundárias (cognitivas). As emoções primárias
surgem a partir de reações a determinadas percepções vindas do meio ambiente e
são simples e geradas rapidamente. Já as emoções secundárias são aprendidas ao
longo da nossa vida, isto é, são lentas e geradas por uma avaliação de uma
situação de acordo com nossos objetivos e valores morais. Por exemplo, André
está com pena de Alberto por possuir determinada doença.

A Computação Afetiva estuda esse assunto dentro da Computação, dividindo-o em
dois ramos principais. O primeiro estuda o reconhecimento e expressão de
emoções dentro da Interação-Homem-Computador (IHC). O segundo estuda a
síntese das mesmas visando estudar o comportamento humano por simulações e,
dessa forma, contribuir para o aprimoramento de seres robóticos ou virtuais.

O presente trabalho visa desenvolver uma ferramenta que permita definir o
comportamento afetivo de atores virtuais. Essa ferramenta é um
\emph{framework} que define as emoções dos atores utilizando ontologias. Uma
ontologia é uma especificação explícita, fundamentada e bem formada de um
conhecimento \cite{gruber1993translation}. Atualmente, elas são vistas como um
entendimento comum e compartilhado que pode ser utilizado na comunicação entre
máquinas ou pessoas \cite{wks2008towards}.

Sendo assim, o principal resultado é a demonstração das três ontologias
funcionando em conjunto. A ontologia de emoções foi definida conforme o modelo
de \citet{ortony1988cse}. A ontologia de rotinas, criada por
\citet{paiva2005ontology}, define a rotina dos personagens pelo perfil destes
e permite descrever os locais que o mesmo pode ir visitar tornando possível
criar uma visualização. Já, definir anotações em objetos e preferências sobre
essas anotações é a tarefa da última ontologia. Ela torna possível construir
um mecanismo para emoções simples relacionadas com objetos.

Este trabalho está organizado da seguinte maneira. No capítulo~\ref{ch:eda}
é feita a revisão bibliográfica das principais áreas desse trabalho:
Agentes, Computação Afetiva e Ontologia. Os trabalhos relacionados também
estão inclusos nesse capítulo em suas respectivas seções. A ideia da
ferramenta é explicada no capítulo~\ref{ch:aec}. Os casos de uso são abordados no
capítulo de estudos de caso. No capítulo~\ref{ch:cf} é feita a avaliação do sistema
juntamente com os trabalhos futuros e considerações finais.
