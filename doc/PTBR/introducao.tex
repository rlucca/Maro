\chapter{Introdução}

Segundo \citet{terzopoulos1998behavioral}, a indústria cinematográfica possui
excelentes exemplos do estudo do comportamento humano. Esse estudo visa imitar
as ações humanas na área de animação. Entretanto, ainda há muito trabalho a
ser feito na construção de um ator virtual que pareça realmente vivo. Um dos
primeiros trabalhos envolvendo emoções, em personagens virtuais, é o de
\citet{bates1994role} visando melhorar a interpretação. Por exemplo, João pode
ser agressivo quando estiver com medo e André fica retraído ao sentir medo.
%Além disso, a emoção pode ser utilizada para guiar o comportamento de
%determinado personagem.

As emoções podem ser categorizadas, segundo \citet{damasio2004erro}, em
primárias (não-cognitivas) e secundárias (cognitivas). As emoções primárias
surgem a partir de reações a determinadas percepções vindas do meio ambiente e
são simples e geradas rapidamente. Já as emoções secundárias são aprendidas ao
longo da nossa vida, isto é, são lentas e geradas por uma avaliação de uma
situação de acordo com nossos objetivos e valores morais. Por exemplo, estou
com pena de Alberto por possuir determinada doença.

A Computação Afetiva estuda esse assunto dentro da Computação, dividindo-o em
dois ramos principais. O primeiro estuda o reconhecimento e expressão de
emoções dentro da Interação-Homem-Computador (IHC). O segundo, estuda a
síntese das mesmas visando estudar o comportamento humano por simulações e,
dessa forma, contribuir para o aprimoramento de seres robóticos ou virtuais.

O presente trabalho visa desenvolver uma ferramenta que permita definir o
comportamento afetivo de atores virtuais utilizando-se de ontologias. Uma
ontologia é uma especificação explícita, fundamentada e bem formada de um
conhecimento \cite{gruber1993translation}. Atualmente, elas são vistas como um
entendimento comum e compartilhado que pode ser utilizado na comunicação entre
máquinas ou pessoas \cite{wks2008towards}.

Este trabalho está organizado da seguinte maneira. No capítulo~\ref{ch:eda}
é feita a revisão bibliográfica dos principais conceitos desse trabalho:
Agentes, Computação Afetiva e Ontologia. Os trabalhos relacionados também
estão inclusos nesse capítulo em suas respectivas seções. A ideia da
ferramenta é explicada no capítulo~\ref{ch:aec}. Os casos de uso são abordados no
capítulo seguinte. No \ref{ch:cf} é feita a avaliação do sistema juntamente
com os trabalhos futuros e considerações finais.
