\chapter{Trabalho proposto}
O desenvolvimento de uma ontologia do modelo afetivo OCC é o foco desse
trabalho. A principal contribuição é a criação de um artefato que possa ser
(re-utilizado) por diferentes sistemas cujos desenvolvedores pretendem
utilizar o modelo afetivo explicado na seção~\ref{CA:2}.  Além disso, durante
a construção da ontologia foi definido não fazer uso de regras exceto aquelas
proporcionadas pela linguagem \emph{OWL}.

Sendo assim, linguagens que permitem estender a \emph{OWL} com regras estão
sendo desconsideradas nessa primeira fase.  Na segunda fase, a plataforma
Jason será adaptada para trabalhar em conjunto com o ambiente desenvolvido.
Esse ambiente disponibilizará anotações junto das percepções do ambiente.
Essas anotações permitem que o objeto informe o que ele é e como deve ser
utilizado.  \citet{doyle1998annotated} introduziram esse uso de anotações como
um jeito do próprio objeto carregar suas informações e permitir que o
conhecimento que outrora estaria no agente esteja espalhado no mundo virtual.

Dessa forma, a definição de um objeto pode conter anotações que visam guiar o
comportamento emotivo do agente que controla um ator. Portanto, o agente
recebe ``dicas'' de como se comportar, podendo ou não segui-las. O
comportamento resultante do ator pode não ser o esperado, porém agentes
semelhantes terem reações diferentes em uma mesma situação é algo que se quer
incentivar.  Esse incentivo é considerado benéfico por parecer um
comportamento próximo do improvisado.

Logo, uma segunda ontologia necessita ser criada para conter o mapeamento
dessas propriedades que virão do ambiente inteligente. Entretanto, uma escolha
nesse momento inicial pode ser prejudicial. A ontologia do modelo de emoções
necessita, inclusive, de dados auxiliares, como, por exemplo, os limites para
uma emoção virar sentimento.

\section{Ontologia}

\section{Plataforma \jason}
