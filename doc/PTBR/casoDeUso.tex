\chapter{Caso de Uso} \label{ch:cdu}

na questao da ontologia de comportamento, acredito q seja possivel
fazer com que um personagem em determinado ciclo vá para um
planeta...

essa questão tras outra pergunta, se um comportamento vai ser
disparado em determinado ciclo para parecer rotina. Quanto tempo,
ou melhor, quantos ciclos representam um dia? E, quanto tempo
é representado por um misero ciclo?

Para responder isso, eu considero o seguinte... eu não vou arredar o
pé de 1000 ciclos. Mas, 1000 estão variando de 4 a 7 minutos reais.
Se considerar que 1000 ciclos possui 10 dias então cada 100 ciclos é um dia.
Dessa forma, usaremos o ciclo equivalente a 870 segundos que correspondem
a 14,5 minutos que equivale a aproximante 99,31 ciclos. Assim, se iniciarmos
o relogio a partir do ciclo 1 no ciclo 100 ou 101 temos uma virada de dia.
Além disso, é importante notar que nem todos os dias viraram no mesmo step.
Por exemplo, alguns vao ter 99 outros 98.... o q torna interessante!

Eu estou pensando em pra ligar com rotinas ter turnos, dividir o dia em
madrugada (24h to 5h30), manha(5h31 to 12h), tarde(12h01 to 18h) e
noite(18h01 to 23h59). Assim, teriamos um turno de 5,5h, outro de 6,5h e os
restantes com 6h.

5 -> Esse tempo pode ser uma percepcao com dia, horario e turno... Simples,
ne?
       E Esse turno `comanda' o que o agente faz no momento porque eh a rotina
       dele, isto eh, pela manha ir em algum lugar amtar gente. Pela tarde
ficar
       matando piratas e a noite/madrugada descansar!

MADRUGADA = M
MANHA = A
TARDE = T
NOITE = N

--
o texto a seguir se refere sempre a ontologia aw.
Nela não temos myself como individuo porque esse
eh um meta individuo para cada ontologia OCC.
Além disso, a alguns atributos novos precisam ser
declarados em ambos. Por exemplo, life que representa
a vida de um personagem. Não existia ate agora,
entretanto a ideia de ter isso visivel como anotacao
pode ser ruim, porém ao inves de life ser visivel aos
outros seria interessante ter visivel damage...
Assim, é interessante pensar no que se quer ver
como exterior e o que se quer ver como myself....
--
Depois disso tem a questão das preferencias ainda,
eu não pensei como nesse nosso ambiente os agentes
vao agir.... eu sei que a maior parte eh mal, gosta de
fazer maldade... alguns são 'companheiros' de outros,
por exemplo podem ver um momento de maldade e
deixar pro colega se divertir...

Outros são excecoes... Vorfield pode ir no planeta
sem matar ninguem, mas a populacao sua não discansa
dessa forma. Ela precisa se afastar... Fora isso o radar dela
 eh melhor com pesos e alcance de 6?

