\documentclass[diss,oneside]{iiufrgs}
%\documentclass[diss,oneside]{iiufrgs}
% um tipo específico de monografia pode ser informado como parâmetro opcional:
%\documentclass[tese]{iiufrgs}
% monografias em inglês devem receber o parâmetro `english':
%\documentclass[diss,english]{iiufrgs}
% a opção `openright' pode ser usada para forçar inícios de capítulos
% em páginas ímpares
% \documentclass[openright]{iiufrgs}
% para gerar uma versão somente-frente, basta utilizar a opção `oneside':
% \documentclass[oneside]{iiufrgs}
\usepackage[T1]{fontenc}        % pacote para conj. de caracteres correto
\usepackage[utf8]{inputenc}     % pacote para acentuação
\usepackage[alf]{abntcite}		% pacote para as referencias da abnt nao numerica
\usepackage{graphicx}           % pacote para importar figuras
\usepackage{times}              % pacote para usar fonte Adobe Times
\usepackage{color}              % pacote para usar \textcolor
\usepackage{xcolor,colortbl}    % pacote para usar cores nas tabelas
\usepackage{listings}       	% pacote para usar listagens/listing
\usepackage{rotating}			% rotacionar textos
\usepackage{multirow}			% textos usando mais de uma linha
%\usepackage{mathptmx}          % p/ usar fonte Adobe Times nas fórmulas
\usepackage{wrapfig}			% posiciona elementos de maneira fluante
\usepackage{url}
\usepackage{setspace}		% para usar onehalfspacing

% Use todo para anotar algo para ver depois
% Criando comando \todo{algo}
\newcommand{\todo}[1] {\footnote{TODO #1}\marginpar{\textcolor{red}{\textbf{TODO}}}}
%\renewcommand{\todo}[1] {}
% Use dev para anotar algum comprometimento
\newcommand{\dev}[0] {\marginpar{\textcolor{blue}{\textbf{IMPL}}}}
\renewcommand{\dev}[0] {}

\newcommand\UP{\mathbin{\char`\^}}

%Citacoes possiveis
%\cite{ortony1988cse} - parenteseses
%\citeonline{ortony1988cse} - texto e parenteses so o ano\\
%\citeauthoronline{ortony1988cse} - so os autores sem ano\\
%\citeyear{ortony1988cse} - so o ano
%\citeauthor{ortony1988cse} - so os autores e sem o 'e'
%atalhos para as citacoes no texto
\newcommand{\citet}{\citeonline}
\newcommand{\jason}{\emph{Jason} }
\newcommand{\occ}{\emph{OCC} }
\newcommand{\OWL}{\emph{OWL} }


% Traduzindo Listing
\renewcommand{\lstlistingname}{Listagem}
\renewcommand{\lstlistlistingname}{Lista de Listagens}

%
% Informações gerais
%
\title{MARO: Um modelo de emoções usando ontologia}

\author{Lucca}{Ricardo Rodrigues}
% alguns documentos podem ter varios autores:
%\author{Flaumann}{Frida Gutenberg}

% orientador e co-orientador são opcionais
\advisor[Prof.~Dr.]{Bordini}{Rafael Heitor}
%\coadvisor[Prof.~Dr.]{Meu}{Seila Ainda}

% a data deve ser a da defesa; se nao especificada, são gerados mes e ano correntes
%\date{fevereiro}{2011}

% o nome do curso pode ser redefinido (ex. para TCs)
%\course{Curso de Especialização em Cachaça}

% o local de realização do trabalho pode ser especificado (ex. para TCs)
% com o comando \location:
%\location{Itaquaquecetuba}{SP}

% itens individuais da nominata podem ser redefinidos com os comandos
% abaixo:
\renewcommand{\nominataReit}{Prof.~Carlos Alexandre Netto}
\renewcommand{\nominataPRCAname}{Vice-Reitor}
\renewcommand{\nominataPRCA}{Prof.~Rui Vicente Oppermann}
\renewcommand{\nominataPRAPG}{Prof.~Aldo Bolten Lucion}
\renewcommand{\nominataDir}{Prof.~Lu{\'i}s da Cunha Lamb}
\renewcommand{\nominataCoord}{Prof.~{\'A}lvaro de Freitas Moreira}

% A seguir são apresentados comandos específicos para alguns
% tipos de documentos.

% Relatório de Pesquisa [rp]:
% \rp{123}             % numero do rp
% \financ{CNPq, CAPES} % orgaos financiadores

% Trabalho Individual [ti]:
% \ti{123}     % numero do TI
% \ti[II]{456} % no caso de ser o segundo TI

% Trabalho de Conclusão [tc]:
% além de definir explicitamente o nome do curso (\course) e o local
% de realização (\location), é necessário redefinir a nominata,
% pois as informações necessárias dependem do curso. Ex.:
%\renewcommand{\nominata}{
%        UNIVERSIDADE FEDERAL DO RIO GRANDE DO SUL\\
%        Reitora: Prof\textsuperscript{a}.~Wrana Maria Panizzi\\
%        Pró-Reitor de Ensino: Prof.~José Carlos Ferraz Hennemann\\
%        Diretor do Instituto de Informática: Prof.~Philippe Olivier Alexandre Navaux\\
%        Coordenador do curso: Prof.~Seu Creysson\\
%        Bibliotecária-chefe do Instituto de Informática: Beatriz Regina Bastos Haro
%}

% Monografias de Especialização [espec]:
% \espec{Redes e Sistemas Distribuídos}      % nome do curso
% \coord[Profa.~Dra.]{Weber}{Taisy da Silva} % coordenador do curso
% \dept{INA}                                 % departamento relacionado

%
% palavras-chave
% iniciar todas com letras minúsculas, exceto no caso de abreviaturas
%
\keyword{agentes virtuais}
\keyword{programação de agentes}
\keyword{simulação}
\keyword{computação afetiva}
\keyword{modelo OCC}
\keyword{ontologia}

%
% inicio do documento
%
\begin{document}

% folha de rosto
% às vezes é necessário redefinir algum comando logo antes de produzir
% a folha de rosto:
% \renewcommand{\coordname}{Coordenadora do Curso}
\maketitle

%\chapter*{Assinaturas}

	\vspace{3cm}

   \begin{center}
        \rule{8cm}{.1mm} \\ Orientador: Prof.~Dr.~Rafael Heitor Bordini
    \end{center}

	\vspace{3cm}

   \begin{center}
        \rule{8cm}{.1mm} \\ Aluno: Ricardo Rodrigues Lucca
    \end{center}

 % Maybe the last stuff or the first?

%% dedicatoria
\clearpage
\begin{flushright}
\mbox{}\vfill
{\sffamily\itshape
``If I have seen farther than others,\\
it is because I stood on the shoulders of giants.''\\}
--- \textsc{Sir~Isaac Newton}
\end{flushright}

%% agradecimentos
\chapter*{Agradecimentos}
Agradeço ao \LaTeX\ por não ter vírus de macro\ldots


%\setcounter{tocdepth}{0} % only chapters
\setcounter{tocdepth}{1} % only chapters and sections
\onehalfspacing %poderia ser double mas nao eh bom

% resumo na língua do documento
\begin{abstract}
O estudo das emoções é feito dentro da computação na área chamada computação
afetiva. Essa área estuda tudo relacionado ou surgido das emoções e o presente
trabalho foca na síntese de emoções utilizando o modelo baseado em
significados proposto por \citet{ortony1988cse} na plataforma \jason
\cite{bordini-jason}. Assim, o desenvolvimento de uma integração entre \jason
e ontologias foi feito. As emoções são guardadas em uma ontologia desenvolvida
e se uma crença for concluída como emoção, ela vira um sentimento quando o
valor atingir um determinado limite mínimo. Dessa forma, a emoção (não
perceptível) passa a ser um sentimento percebido.
\end{abstract}

% resumo na outra língua
% como parametros devem ser passados o titulo e as palavras-chave
% na outra língua, separadas por vírgulas
\begin{englishabstract}{Maro: An emotional model using ontology}{virtual agents, programming of
agents, simulation, affective computing, OCC's model, ontology}
The study of emotions is made in computing at the field called affective
computing. That field studies all things related or emerged from emotions. The
present work is focused on emotional synthesis using a meaning-based emotion
model \cite{ortony1988cse} at \jason platform \cite{bordini-jason}. Thus, the
development of a integration between \jason and ontology was created. All
emotions are saved in an ontology development. Therefore, the ontology is used
to classify the belief of an agent as emotion and when that emotion hits a
threshold it becomes a feeling. In that way, the emotion (not perceived)
passes to be a perceived feeling.
\end{englishabstract}

 % inclui o resumo da lingua nativa e estrangeira
\singlespacing %poderia ser double mas nao eh bom
\tableofcontents % sumario
\onehalfspacing %poderia ser double mas nao eh bom
% lista de abreviaturas e siglas
% o parametro deve ser a abreviatura mais longa
\begin{listofabbrv}{IHC}
		\item[AOP] \underline{A}gent \underline{O}riented \underline{P}rogramming
		\item[ASL] \underline{A}gent\underline{S}peak \underline{L}anguage
		\item[BDI] \underline{B}elief-\underline{D}esire-\underline{I}ntention
        \item[IHC] \underline{I}ntera\c{c}\~{a}o \underline{H}omem-\underline{C}omputador
        \item[OCC] \underline{O}rtony \underline{C}lore e \underline{C}ollins
		\item[OWL] \underline{O}ntology \underline{W}eb \underline{L}anguage
		\item[UEM] \underline{U}rban \underline{E}nvironment \underline{M}odel
        \item[UML] \underline{U}nified \underline{M}odeling \underline{L}anguage
        \item[XML] e\underline{X}tensible \underline{M}arkup \underline{L}anguage

\end{listofabbrv}



% LISTA DE SIMBOLOS
% o parametro deve ser o simbolo mais longo
%\begin{listofsymbols}{$\alpha\beta\pi\omega$}
%       \item[$\sum{\frac{a}{b}}$] Somatório do produtório
%       \item[$\alpha\beta\pi\omega$] Fator de inconstância do resultado
%\end{listofsymbols}

\listoffigures % lista de figuras, se houver mais de 3
%\listoftables % lista de tabelas, se houver mais de 3
\lstlistoflistings % lista de listagens, se houver mais de 3

\definecolor{darkgray}{rgb}{0.95,0.95,0.95}
\lstset{backgroundcolor=\color{darkgray}}
\lstset{numbers=left, basicstyle=\scriptsize\ttfamily, numberstyle=\tiny, stepnumber=2, numbersep=5pt}


%% aqui comeca o texto propriamente dito
\chapter{Introdução}

Segundo \citet{terzopoulos1998behavioral}, a indústria cinematográfica possui
excelentes exemplos do estudo do comportamento humano na área de animação.
Essa área visa imitar as ações humanas computacionalmente. Entretanto, ainda
há muito trabalho a ser feito na construção de um ator virtual que pareça
realmente vivo. Um dos primeiros trabalhos envolvendo emoções, em personagens
virtuais, é o de \citet{bates1994role} visando melhorar a interpretação. Por
exemplo, João pode ser agressivo quando estiver com medo e André fica retraído
ao sentir medo.

As emoções podem ser categorizadas, segundo \citet{damasio2004erro}, em
primárias (não-cognitivas) e secundárias (cognitivas). As emoções primárias
surgem a partir de reações a determinadas percepções vindas do meio ambiente e
são simples e geradas rapidamente. Já as emoções secundárias são aprendidas ao
longo da nossa vida, isto é, são lentas e geradas por uma avaliação de uma
situação de acordo com nossos objetivos e valores morais. Por exemplo, André
está com pena de Alberto por possuir determinada doença.

A Computação Afetiva estuda esse assunto dentro da Computação, dividindo-o em
dois ramos principais. O primeiro estuda o reconhecimento e expressão de
emoções dentro da Interação-Homem-Computador (IHC). O segundo, estuda a
síntese das mesmas visando estudar o comportamento humano por simulações e,
dessa forma, contribuir para o aprimoramento de seres robóticos ou virtuais.

O presente trabalho visa desenvolver uma ferramenta que permita definir o
comportamento afetivo de atores virtuais. Essa ferramenta é um
\emph{framework} que define as emoções dos atores utilizando ontologias. Uma
ontologia é uma especificação explícita, fundamentada e bem formada de um
conhecimento \cite{gruber1993translation}. Atualmente, elas são vistas como um
entendimento comum e compartilhado que pode ser utilizado na comunicação entre
máquinas ou pessoas \cite{wks2008towards}.

Sendo assim, o principal resultado é a demonstração das três ontologias
funcionando em conjunto. A ontologia de emoções foi definida conforme o modelo
de \citet{ortony1988cse}. A ontologia de rotinas, criada por
\citet{paiva2005ontology}, define a rotina dos personagens pelo perfil destes
e permite descrever os locais que o mesmo pode ir visitar tornando possível
criar uma visualização. Já, definir anotações em objetos e preferências sobre
essas anotações é a tarefa da última ontologia. Ela torna possível construir
um mecanismo para emoções simples relacionadas com objetos.

Este trabalho está organizado da seguinte maneira. No capítulo~\ref{ch:eda}
é feita a revisão bibliográfica das principais áreas desse trabalho:
Agentes, Computação Afetiva e Ontologia. Os trabalhos relacionados também
estão inclusos nesse capítulo em suas respectivas seções. A ideia da
ferramenta é explicada no capítulo~\ref{ch:aec}. Os casos de uso são abordados no
capítulo seguinte. No capítulo~\ref{ch:cf} é feita a avaliação do sistema
juntamente com os trabalhos futuros e considerações finais.

\chapter{Estado da Arte} \label{ch:eda}
\section{Programação Orientada a Agentes}

\todo{rever}
%A presente seção se encontra divida em uma introdução a programação de
%agentes na seção~\ref{AOP}, onde é discutido o paradigma e os termos agentes
%e personagens. Na seção~\ref{CA:1} a área de computação afetiva é
%apresentada.  O conceito de ontologia é debatido na seção~\ref{onto} e os
%trabalhos relacionados são apresentados na seção~\ref{TR}.

\subsection{Visões Diferentes sobre o Conceito Agente}

\citet{laird2001human} afirmam que a pesquisa em Inteligência Artificial tem
sido fragmentada em muitas áreas especializadas e, assim, algoritmos
específicos e mais eficientes podem ser criados.
%
Há várias definições do conceito de Agente, por exemplo
\cite{shoham1993agent,roadmap,fatemeh2009multi}. \citet{shoham1993agent}
propõe que Agente é uma entidade definida por componentes mentais como
crenças, habilidades, escolhas e comprometimentos.

\citet{franklin1997agent} define um agente através das características que a
entidade deve apresentar.  Atualmente, o menor conjunto comumente aceito
dessas características é composto por três conceitos principais: (i)
Autonomia; (ii) Sociabilidade; (iii) Situacionalidade.
\citet{roadmap,fatemeh2009multi} concordam que autonomia é quando um Agente
toma suas próprias decisões independente de qualquer outra entidade do sistema
ou, ainda, da intervenção diretas de seres humanos. A característica de
sociabilidade permite flexibilidade na execução das tarefas, através da
interação com outros agentes que estejam presentes no sistema. A última,
permite que o agente situe-se em um ambiente dinâmico interagindo com o mesmo
através de algum tipo de sensor ou atuador.

\citet{ingrand1992architecture} indicam que entidades autônomas podem ter a
capacidade de expor seus dados internos para que seja possível um usuário,
possivelmente humano, dar dicas sobre a forma de resolução dos problemas sendo
enfrentados. Esta visão não conflita com a noção de autonomia exposta acima,
pois o agente permanece independente para tomar suas decisões podendo rejeitar
as dicas ou sugestões enviadas pelo usuário.

\citeauthor{ingrand1992architecture} também define Agente em um ambiente com
componentes heterogêneos e com diferentes tempos de resposta para a execução
de suas tarefas. \citet{doyle1998annotated} estendem o conceito dizendo que os
objetos pertencentes ao ambiente devem conter anotações que determinam a forma
de uso dos objetos disponibilizados neles. Assim, não se sabe como todos os
objetos funcionam e, sim, uma forma de aprender com o próprio objeto a sua
forma de utilização. \citet{shoham1993agent} define sociabilidade como uma
habilidade cognitiva necessária para o desempenho das suas tarefas.

Essas inúmeras definições do termo Agente não permite saber se ele é um ser
físico ou abstrato. Dessa forma, \citet{nareyek2001review,damiano2008emotions}
defendem que o agente é o ser abstrato de um ator físico. Em outras palavras,
o ser que age ou atua no meio é chamado de personagem ou ator. Enquanto a
mente desse ser é chamada de agente.


\subsection{Paradigma de Programação} \label{sec:aoppp}

\citet{shoham1993agent} propôs em seu trabalho a linguagem \emph{Agent-0} uma
das primeiras a serem baseadas em atos de fala e uma nova visão para se
programar.  Esses atos de fala podem ser vistos como comandos falados entre
atores com as mais diversas finalidades (informar, perguntar, requerer,
aceitar, etc).  Dessa forma, \citeauthoronline{shoham1993agent} tentou promover a
idéia de computação com uma interação mais social entre membros de um sistema.
Em outras palavras, esse novo paradigma de programação precisa que exista
cooperação e competição entre os agentes para a realização das tarefas
desejadas.

Assim, o agente encontra-se situado em um ambiente no qual pode receber de ou
enviar para outros agentes informações. As ações são
estabelecidas utilizando regras que descrevem comportamentos. Logo, é
possível, a própria entidade decidir qual ação deve ser tomada. Essas regras
são construídas em fórmulas lógicas descrevendo-se o contexto que torna um
curso de ação válido e sua consequência. Essa consequência pode ser uma ação
ou um conjunto sequencial de ações que precisam ser feitos.

Atualmente, segundo \citet{bordini2009multi}, há diversas linguagens para a
AOP. Por exemplo: \emph{2APL}, \emph{Agent-0}, \emph{AgentSpeak(L)},
\emph{GOAL} e \emph{MetateM}. Elas são baseadas em diversos
formalismos diferentes. \emph{MetateM}, por exemplo, é baseada em lógica temporal,
permitindo que formulas temporais definam o comportamento do agente.
\emph{GOAL} tem a visão que o conhecimento pode ser estático (imutável) ou
dinâmico. \emph{Agent-0}, como dito anteriormente, é a primeira linguagem da
área de programação orientada a agente. \emph{2APL} dividiu a linguagem em
dois formatos básicos \cite{dastani20082apl}. O primeiro define quantos e em
qual ambiente o agente se encontra e o segundo formato define a programação
propriamente dita do indivíduo.

No presente trabalho, a plataforma \jason \cite{bordini-jason} será utilizada.
Ela usa uma extensão da linguagem \emph{AgentSpeak(L)}, definida por
\citet{rao1996agentspeak}, para especificar as atitudes que o agente deve
tomar frente as constantes modificações do ambiente. Assim, essa linguagem é,
normalmente, vista como reativa as crenças, metas ou percepções.


\subsection{Plataforma \jason} \label{sec:aoppj}

A presente seção visa explicar a plataforma \jason baseada na arquitetura BDI.
Essa arquitetura é baseada no que o agente sabe ou guarda (crenças), as opções
que ele possui de atuação baseado em suas próprias regras (desejos) e seus
comprometimentos (intenções). Assim, na seção \ref{sec-jason-overview} uma
visão geral da plataforma é apresentada e em seguida a engenharia da mesma é
discutida.

\subsubsection{Visão Geral} \label{sec-jason-overview}

Para uma explicação mais didática será utilizado o exemplo \emph{Room} que
acompanha o \emph{Jason}. Para isso será introduzido
o arquivo de projeto do exemplo e, a partir desse, os demais arquivos.
A Listagem~\ref{lst-roomMas2j} contém o arquivo de projeto.
Na linha 2 da listagem, \emph{room} é o nome do projeto e, por isso, pode ser qualquer
identificador. A partir da linha 3 os valores antes dos dois-pontos (:)
são palavras reservadas que o \jason entende para diferentes propósitos e o valor
a seguir (depois dos dois-pontos) é o valor associado.
Assim, \emph{infrastructure}, na linha 3, pode assumir três valores
possíveis: (i) \emph{Centralised}, normalmente utilizada;
(ii) \emph{Jade}, utilizada quando se deseja integrar
com agentes não \jason (jade) ou rodar distribuído na rede; (iii) \emph{Saci}, utilizada
quando deseja executar os agentes de maneira distribuída na rede.

\begin{center}
    \begin{minipage}{120mm}
	\lstset{linewidth=120mm}
	\begin{lstlisting}[frame=trbl, caption=Arquivo de projeto do \jason para o exemplo \emph{Room}, label=lst-roomMas2j]
// Isso eh um comentario
MAS room {
  infrastructure: Centralised
  environment: RoomEnv
  executionControl: jason.control.ExecutionControl
  agents: porter; claustrophobe; paranoid;
}
	\end{lstlisting}
    \end{minipage}
\end{center}

Continuando na Listagem~\ref{lst-roomMas2j}, a entrada \emph{environment}
configura a classe de ambiente que será utilizada. A próxima entrada,
normalmente não aparece, é a \emph{executionControl} utilizada para
mudar a forma com que os agentes são executados. O valor no exemplo é uma
classe que obriga o próximo ciclo de deliberação acontecer somente quando
todos os agentes terminarem o seu ciclo. O padrão é iniciar um novo
após 500ms de o anterior ter sido concluído, porém, algumas vezes isso
pode vir a apresentar problemas de sincronismo por ser assíncrono e, por isso, é interessante
mostrar que há uma opção para controlar a forma de execução dos ciclos
deliberativos. O usuário, inclusive, pode colocar sua própria classe como
configuração.

Todas as entradas apresentadas até agora mapeiam para código Java.
Os agentes são especificados na entrada \emph{agents}. Como se pode
observar na Listagem~\ref{lst-roomMas2j}, cada referência a um agente deve
terminar com um ponto-e-vírgula (;). Há ainda como especializar cada um dos
agentes mudando opções. Na seção~\ref{sec-jason-architecture} e no
capítulo~\ref{ch:cdu} são mostrados exemplos dessas opções.

Os agentes são desenvolvidos em arquivos texto com extensão \emph{ASL}, porém
antes de entrar em discussão sobre os agentes será explicado o exemplo sendo
utilizado. Nesse cenário tem-se uma porta na sala e duas pessoas, uma
claustrofóbica e outra paranoica. A pessoa claustrofóbica deseja que a
porta da sala esteja aberta, enquanto que a paranoica deseja que a porta
fique fechada.

Logo, há três agentes na linha 6 da Listagem~\ref{lst-roomMas2j}:
(i) \emph{porter}, o agente responsável pela porta e o único que conhece
como abri-lá ou fecha-lá;
(ii) \emph{claustrophobe}, o agente que deseja deixar a porta sempre aberta;
(iii) \emph{paranoid}, o agente que deseja deixar a porta sempre fechada.
A implementação desses agentes faz uso de eventos.

Esses eventos podem ser de adição (+) ou de remoção (-) de crenças, metas ou
consultas. Todas as estruturas são semelhantes a chamada de função. No
exemplo ``telefone(808080822)'', telefone pode ser uma crença ou uma ação de
ambiente que pode ser executada dependendo de onde a expressão aparece.
Note que, se essa entrada for precedida pelo sinal de exclamação (!) ou de
interrogação (?) então o significado é alterado para uma meta ou para uma
consulta respectivamente. Ainda há a possibilidade de preceder uma crença ou
meta com sinal de adição ou de subtração significando ou estar
adicionando/removendo uma crença ou estar adicionando/removendo uma meta ou
consulta.

\lstset{linewidth=75mm}
\begin{wrapfigure}{l}{85mm}
	\begin{lstlisting}[frame=trbl, caption=Agentes em ASL, label=lst-agente]
// claustrophobe.asl
+locked(door) : true
  <- .send(porter,achieve,~locked(door)).

// paranoid.asl
+~locked(door)
  <- .send(porter,achieve,locked(door)).

// porter.asl
+!locked(door)[source(paranoid)]
  : ~locked(door)
  <- lock.

+!~locked(door)[source(claustrophobe)]
  : locked(door)
  <- unlock.
	\end{lstlisting}
\end{wrapfigure}
%
Na Listagem~\ref{lst-agente} tem-se a continuação do exemplo e, por
simplicidade, todos os fontes dos agentes encontram-se reunidos. O agente denominado
\emph{claustrophobe} será o primeiro a ser detalhado e corresponde as linhas 1 até 3
da Listagem~\ref{lst-agente}. A programação em \jason é
guiada por reatividade às crenças e percepções do próprio agente, assim, é necessário
uma forma de estruturar as ações à serem decididas. Essa forma é o plano.
O plano pode ser ativado quando se deseja adicionar, consultar ou remover
\footnote{Uma remoção pode ser gerada devido a uma falha, porém isso não será
abordado.} uma crença ou meta. O plano da linha 2 até 3 da
Listagem~\ref{lst-agente} será detalhado adiante.

Em um plano há sempre três divisões: evento disparador, contexto e corpo.
A primeira e única à ser explicita é o evento disparador, que deve ocorrer para
o plano ser disparado. Ele está limitado do caractere inicial até
o dois-pontos (:). A divisão de contexto é onde se coloca as ações, crenças
ou regras que tem que ser válidas para o corpo do plano ser
considerado válido. Dessa forma, é
importante tomar cuidado no que vai no contexto em razão dele sempre ser
executado previamente para definir quais planos devem ser descartados da lista
de opções.
A segunda divisão vai até a seta (<-) e não é obrigatória.
A terceira divisão, também não é obrigatória, possui todas as ações a serem
executadas quando o plano tornar-se ativo.

No exemplo o plano tem em seu corpo
uma ação interna do \jason denominada \emph{send} que envia determinada
mensagem
(terceiro parâmetro) para o agente especificado (primeiro parâmetro) com
determinado propósito (segundo parâmetro). Uma ação interna possui o seguinte
formato ``tcp.send'', assim essa ação está definida no pacote \emph{tcp} pela
classe \emph{send} que deve especializar a classe \emph{DefaultInternalAction}
do \emph{Jason}. Entretanto, uma ação interna definida pelo \jason não possui
indicativo de pacote que é o caso do ``.send'' presente no código dos agentes.

No exemplo da linha 3 na Listagem~\ref{lst-agente}, o agente
\emph{claustrophobe} está enviando uma mensagem para o agente denominado
\emph{porter} ter a meta (\emph{achieve} no segundo parâmetro) de ter a crença
que a porta não ($\sim$) está fechada. Analogamente, o agente denominado
\emph{paranoid}, definido na linha 5 à 7, envia como crença ter a porta
fechada. Logo, o agente \emph{porter} pode ser entendido em sua quase
totalidade na Listagem~\ref{lst-agente}. Um entendimento completo do exemplo
vem com a compreensão das anotações que são informações que podem ser
guardadas junto das crenças e essas anotações podem ter suas próprias
anotações também. No agente \emph{porter} essas anotações são utilizadas
somente para dizer que determinado plano só é válido quando tiver como fonte
(do inglês \emph{source}) um determinado agente. O presente exemplo assume a
hipótese do mundo aberto. Vale observar que essas anotações podem ser
removidas sem nenhum problema adicional, entretanto, se isso for feito
assume-se a hipótese do mundo fechado e não seria possível ter um novo agente
que o \emph{porter} ignora.

A execução dos agentes acontece de forma arbitrária e deve-se ter em conta
que o primeiro agente que irá enviar a mensagem para o agente \emph{porter} dependerá
de como o mundo inicia. Logo, se o mundo iniciar com a porta fechada o primeiro
a enviar a solicitação será o agente \emph{claustrophobe}. Já se o mundo for iniciado
com a porta aberta o primeiro a enviar solicitação será o agente \emph{paranoid}.
Depois disso, conforme a simulação vai correndo, os dois agentes ficam alternando
mensagens com o agente \emph{porter} por causa do compartilhamento do mesmo
recurso (a porta).

\vfill

\subsubsection{Infra-estrutura do \jason} \label{sec-jason-architecture}

A plataforma \jason tem uma base de software completamente extensível como
é possível observar mediante a Figura \ref{fig-jason-infra-1}.
Nessa figura, a infraestrutura encontra-se representada como um barramento
na parte inferior. Ela pode ser configurada através da chave de configuração
\emph{infrastructure} no arquivo de projeto (ver Listagem~\ref{lst-roomMas2j}).

A infraestrutura define todas as classes padrões que o usuário irá utilizar,
por exemplo observe que o barramento liga-se a dois adaptadores: um do tipo
agente e um de ambiente. Esses adaptadores permitem a compatibilidade entre
implementações distintas permitindo que ocorram variações sem ter que alterar
a estrutura do barramento.

\begin{figure}
               \begin{center}
               \includegraphics[width=140mm]{figuras/infra.png} 
                \end{center}
                \caption{Modelo da infraestrutura do \emph{Jason}.}
                \label{fig-jason-infra-1}
\end{figure}

Assim, há a possibilidade de informar para uma determinada simulação que um
determinado agente utilizará uma arquitetura e/ou um raciocinador diferente
dos demais agentes. Para se fazer isso no momento que se declara os agentes
no projeto informa-se as chaves que encontram-se dentro das caixas com
cantos arqueados na Figura~\ref{fig-jason-infra-1}. O exemplo
``fb agArchClass KosMos.FireBrigadeArch agClass KosMos.Agent \#1;'' define
um (\#1) agente \emph{fb} com a arquitetura usando a classe \emph{FireBrigadeArch}
do pacote \emph{KosMos} (pode ser qualquer nome desejado) e configurando o
raciocinador a partir da classe \emph{Agent} do mesmo pacote. Note que, essas classes
não existem no \jason e devem ser providas pelo usuário de alguma forma.

Como já explicado, a entrada \emph{environment} no arquivo de projeto
configura a classe de ambiente que será utilizada e somente um
é permitido por simulação. Um dos motivos comuns de se implementar um
ambiente é implementar as ações que o agente poderá realizar no mesmo. Aliás,
também, é possível alterar como os agentes percebem o meio para, por
exemplo, inserir percepções incorretas em alguns momentos.

Essa infraestrutura extensível permitiu a implementação do trabalho de
\citet{moreira2006agent} por \citet{KlaBor09} sem nenhuma alteração profunda
nas classes da plataforma. Esse trabalho permitiu que o \jason usa-se
ontologias, inclusive definindo conceitos e compartilhando o conhecimento. Por
exemplo, um agente pode dizer para outro que a informação que ele deseja
encontra-se em determinada ontologia. Além disso, há outras extensões como
não usar arquivos de projeto e sim uma base de conhecimento baseada em
ontologia para definir a simulação.

O presente trabalho segue uma ideia parecida desses dois anteriores. A
principal preocupação era ter as crenças que a ontologia considere relevantes
em sua base de conhecimento e as outras guardadas na base padrão. Isso não é
diferente no trabalho anterior, o que é diferente é o como a relevância é
verificada. No trabalho anterior, o que era salvo na base era tudo que estivesse com a
anotação usada para indicar a ontologia e no presente desenvolvimento isso é
extraído da própria. Assim, se na ontologia contiver
o conceito Pizza e inserimos uma crença ``pizza(p1)''. O mecanismo
desenvolvido sabe que essa crença é relevante para a ontologia. Já o do
trabalho anterior para saber que o mesmo é relevante precisa ser marcado
com uma anotação, por exemplo ``pizza(p1)[ontology(URL)]''. Além disso, ele
não lida com emoções e no presente trabalho os agentes podem perceber seus
sentimentos.




\section{Computação Afetiva}

\citet{Pic98} definiu Computação Afetiva como uma ``computação relacionada,
surgida ou que influência as emoções''. Além disso, computadores com emoções
permitem aos mesmos um determinado nível de comportamento inteligente e
criatividade que seria impossível sem as emoções e esse é o principal desafio
dessa área. Logo, o seu entendimento pode explicar fenômenos como, por
exemplo, atenção, memória e outros.


Essa área é normalmente dividida em duas sub-áreas. A primeira estuda o
reconhecimento e a expressão de emoções dentro da Interação Homem-Computador;
a segunda, foca na síntese de emoções para aprimorar os seres robóticos e/ou
para estudar o comportamento humano por meio de simulações. Há muita
aplicabilidade dessas técnicas, por exemplo: a área que reconhece as emoções
pode ser utilizada para adaptar o sistema ao estado da pessoa permitindo ao
mesmo instruí-la, questioná-la, encorajá-la ou ocultar determinadas
informações consideradas irrelevantes.

O objetivo de \citet{bick2003relational} com o projeto \emph{Relational
Agents} é possibilitar aos usuários a criação de um relacionamento social e
emocional com longa duração.  Em \citet{bickmore2009virtual}, a confiança no
agente torna possível discutir tarefas mais importantes como melhoria da saúde
ou até a compra de uma casa. Outro trabalho na área de IHC é o reconhecimento
de emoções para aumentar a imersão em jogos, por exemplo permitindo ao próprio
jogo adaptar eventos ou trechos tornando-o mais divertido e realista.

\begin{figure}
  \begin{center}
    \includegraphics[width=75mm]{figuras/tigger-mit.png}
  \end{center}
  \caption{Brinquedo que responde as emoções das crianças \cite{kirsch1999affective}.}
  \label{fig:tigger-mit}
\end{figure}

O projeto \emph{The Affective Tigger: a reactive expressive toy} de
\citet{kirsch1999affective} é um brinquedo capaz de reconhecer e reagir às
emoções exibidas pelas crianças. Por exemplo, quando a criança encontra-se
feliz, o boneco expressa felicidade (ver Figura~\ref{fig:tigger-mit}). Ao todo
existem 5 estados emocionais: muito feliz, feliz, neutro, triste e muito
triste. Todos, com exceção do neutro, possuem alguma síntese vocal como um
rosnado (tristeza) ou uma risada (muito feliz). Assim, esse brinquedo, por ser
considerado um ser robótico que reage à criança com seus próprios estados
emocionais, fica enquadrado na segunda área.  Portanto, o desenvolvimento
desse brinquedo serviu para aprimorar os seres robóticos.

O projeto AIDA\footnote{Mais detalhes, ver http://senseable.mit.edu/aida} (do
inglês \emph{Affective Intelligent Driving Agent}) pode ser entendido como
enquadrado na área de IHC, pois o interesse é entender o estado afetivo da
pessoa dirigindo. Além disso, interessa-se em ter um relacionamento com o
usuário sugerindo alterações nas rotas baseado na rotina aprendida depois de
um mês de aprendizado.  A pesquisa relatada em \citet{dias-agents} visou
melhorar a simulação de agentes através do uso da emoção guiando o processo
deliberativo e melhorar o entendimento e gerência das emoções.  O presente
trabalho se enquadra na área de síntese de emoção, pois o interesse é em
entender o estado emocional e como ele pode afetar o comportamento de um
personagem.

\subsection{Modelo Cognitivo Emocional} % TI 2
\subsection{Ontologias Afetivas} % TI 2
\subsection{Arquiteturas Emocionais} % sao so as similares (usam o OCC)
\subsubsection{O Raciocinador Afetivo} % June 1992
% At least 2 pages
\subsubsection{Conjunto de Ferramentas Em} % May 1996 bates-based
% At least 2 pages
\subsubsection{Émile} % L? 2000
% At least 2 pages
\subsubsection{EMA} % P? 2004
% At least 2 pages
\subsubsection{ALMA} %K? 2005
% At least 2 pages
\subsubsection{FATIMA/fearNot!} % X? 2009/ Y? 2007
% At least 2 pages
\subsubsection{WASABI} % March? 2008
% At least 2 pages


\section{Trabalhos Relacionados}
%\section{Ontologias de Humanos Virtuais ou Similares} % TI 2
%eu nao me lembro o que eh isso comentado, talvez nao seja uma
%secao seja so uma nota para ir la ver e entao botar aqui

\todo{eu acho que isso aqui ta muito relacionado com Agents, cade a ontologia? ou nao precisa?}

O comportamento emotivo do personagem tem um papel importante para se ter a
ilusão de vida conforme afirmou \citet{bates1994role}. Esse trabalho utiliza o
modelo OCC para melhorar a credibilidade dos agentes e cada uma das emoções
pode ser ligada a um comportamento. No exemplo dado anteriormente, um agente
pode ligar o medo a um comportamento agressivo enquanto outro liga essa mesma
emoção de medo a um comportamento de estado alarmado.

\citet{GraCli98} criaram um mecanismo evolucionário utilizando redes neurais
com uma base química para guiar o comportamento do ator. Os atores simulados
podem envelhecer, aprender e, inclusive, se reproduzir (aqui são utilizados
algoritmos genéticos).  Por exemplo, o personagem pode aprender algumas
palavras básicas e demostrar que esta envelhecendo por meio da mudança da cor
de seus cabelos dando uma certa ilusão de vida.

Conforme já dito, as emoções podem melhorar a credibilidade dos seres
virtuais.  \citet{zhang2009emotional} desenvolveram uma aplicação com a
finalidade de demostrar esse conceito. O planejamento das ações a serem
executadas pelo personagem é afetado pelos valores das emoções sendo
experimentadas.

Todos os trabalhos apresentados até aqui, utilizaram as emoções para aumentar
a representação ou expressividade de um ator. Entretanto,
\citet{neto2010construction} tentam estudar o impacto da emoção na decisão dos
agentes.  Assim, para isso ser possível, foi necessário alterar a forma de
planejamento das decisões, além de como recuperar os dados da sua memória.  A
modificação no acesso a memória permite o ``esquecimento'' de determinadas
crenças quando o estado emocional for diferente daquele guardado junto da
memória a ser recuperada. Essa característica torna o planejamento do agente e
as atitudes dos personagens mais realísticas.

\citet{benta2007ontology} e \citet{wks2008towards} descrevem modelos afetivos
através de ontologias. No primeiro trabalho, uma ontologia foi criada
descrevendo emoções primárias e secundárias. O primeiro tipo de emoção exige
menos processamento cognitivo que o segundo. Além disso, o modelo descreve ao
todo 11 emoções sendo 7 dessas consideradas primárias.




\chapter{Arquitetura Emocional Constru\'ida} \label{ch:aec}

\todo{talvez botar umas figuras introdutorias aqui ou na proxima secao...}

\section{Visão Geral}
\section{Ontologia do Modelo Cognitivo Emocional} % TI2
\section{Integrando a Plataforma Jason com a Ontologia}

\chapter{Caso de Uso} \label{ch:cdu}

O presente capítulo visa explicar o caso de uso da ontologia. O caso de uso da
ontologia é uma simulação de uma casa em sua vida cotidiana normal, isto é, os
personagens vão para escola, trabalho, fazem compras e ficam em casa.
Entretanto, antes desse caso de uso ser abordado na seção~\ref{ch:cdu:svc}
será falado como a ontologia foi testada juntamente com a plataforma \jason.
A seção~\ref{ch:cdu:tbc} explica as ferramentas desenvolvidas para testar a base
de crenças e, consequentemente, a ontologia. Elas são duas uma interativa e
outra não interativa que serve como uma especie de teste unitário.

\section{Teste da Base de Crenças} \label{ch:cdu:tbc}

Os testes da base de crenças tem como finalidade checar se a utilização normal
esta acontecendo da maneira esperada. Assim, eles fazem testes de inserção,
recuperação, remoção e listagem. A listagem é feita implementando a interface
\emph{Iterable} da plataforma Java e é usada, principalmente, quando a
interface gráfica da plataforma \jason esta em modo depuração e
exibindo a base de crenças.

\begin{figure}
	\begin{center}
		\includegraphics[width=70mm]{figuras/introductionDF.png}
	\end{center}
	\caption{Interface para mostrar os sentimentos dos agentes.}
	\label{fig:introducaoDF}
\end{figure}

A primeira aplicação de teste desenvolvida foi de maneira interativa conforme
pode ser observado na Figura~\ref{fig:introducaoDF}. A aplicação permite que o
usuário escreva as crenças do agente em um campo texto (área 2 na figura) e
esse é enviado diretamente para a base de crença do agente. A área 1 na mesma
figura é utilizada para demostrar a valência das emoções.
Essas informações podem aparecer em: preto, se não houve alteração com o ciclo
anterior; azul, se houve um aumento; vermelho, se houve uma diminuição do
valor.

\begin{figure}
	\begin{center}
		\includegraphics[width=150mm]{figuras/beforeLastInsertionOfPride.png}
		\includegraphics[width=150mm]{figuras/afterLastInsertionOfPride.png}
	\end{center}
	\caption{Exemplo de utilização criando uma emoção de orgulho.}
	\label{fig:testeJasonIntBase}
\end{figure}

Na Figura~\ref{fig:testeJasonIntBase} estão representados dois momentos da
criação de uma emoção de orgulho da parte do usuário. Na parte de cima, o
agente tem configurado uma avaliação de e sobre si próprio. Além disso, a
avaliação tem probabilidade nula ou irrelevante e o desejo para si mesmo foi
considerado com o valor 10. Esse valor poderia ser qualquer número inteiro,
mas se fosse negativo ao invés de alegria (\emph{joy}) o resultado atual seria
sofrimento (\emph{distress}). No campo de texto, o usuário irá inserir a
última relação para ser construída a percepção de orgulho.

Na parte de baixo da Figura~\ref{fig:testeJasonIntBase} pode se ver o
resultado obtido para a inserção da última crença. Conforme pode ser visto,
foi necessário três ciclos deliberativos para o agente ter a percepção do
sentimento. Na verdade, essa percepção não veio de um novo ciclo de simulação
rodando e sim que o agente acrescentou uma nova crença dizendo que havia um
passo novo então foi feita a avaliação das emoções quanto a necessidade de
se criar a percepção afetiva. Se fosse um passo normal da simulação, não se
teria na base de crenças duas percepções \emph{feeling} sobre uma emoção
porque a todo novo passo as percepções são limpas.

Na Listagem~\ref{lst:testeJasonIntBase} pode ser observado uma configuração
para se rodar a apresente aplicação\footnote{Veja a seção
\ref{sec-jason-overview} na página~\pageref{sec-jason-overview}
para ver uma introdução sobre esse tipo de arquivo.}. O ambiente na linha 4
espera que todos os agentes (configurados inicialmente no parâmetro 3) mandem
uma ação para prosseguir. O simulador espera pelo tempo (em milissegundos)
configurado no primeiro parâmetro por essas ações, caso não venha ignora o
agente e segue para o próximo ciclo. O segundo parâmetro permite que seja
configura um número que representa o último passo de simulação e no quarto
parâmetro permite configurar se uma segunda ação enviada no mesmo passo de
simulação é para ser enfileirada ou resultar em erro (essa é a opção atual).

\begin{center}
    \begin{minipage}{130mm}
	\lstset{linewidth=130mm}
	\lstinputlisting[frame=trbl, caption=Arquivo de projeto do \jason para a
aplicação interativa de teste, label=lst:testeJasonIntBase]{../../sampletConsole/eoaus.mas2j}
    \end{minipage}
\end{center}

A agente millie configurada da linha 7 à 11 utiliza opções do \jason para
alterar a base de crença sendo utilizada (\emph{beliefBaseClass}),
arquitetura do agente (\emph{agentArchClass}) e o próprio agente
(\emph{agentClass}). Dessas alterações, a mudança da base de crença e da
classe do agente foram explicadas na seção~\ref{ch:p:ipjo}\todo{garantir isso
depois}. Assim, para o presente exemplo a mudança da arquitetura na linha 9
foi realizada para criar e atualizar a janela que exibe os dados emotivos.

Cabe chamar a atenção que na linha 8 da Listagem~\ref{lst:testeJasonIntBase}
foi usada a ontologia afetiva desenvolvida somente. Assim, para o agente
utilizar as emoções ainda é necessário incluir as configurações do valor
excedido para se ter a emoção presente via código do agente. Uma amostra do
código utilizado para se fazer isso pode ser vista na
Listagem~\ref{lst:testeJasonIntSetup} e todo o código pode ser visto no
Anexo~X\todo{referencia e pagina}.

\lstset{linewidth=80mm}
\begin{wrapfigure}{l}{90mm}
	\begin{lstlisting}[frame=trbl,
caption=Parte do código do agente para aplicação interativa de teste,
label=lst:testeJasonIntSetup]
step(0)[source(percept),source(self)].

agent(millie).

hasSetup(millie, setup1).
hasThreshold(setup1, 0).
hasThresholdType(setup1, "Joy").

hasSetup(millie, setup2).
hasThreshold(setup2, 0).
hasThresholdType(setup2, "Distress").
	\end{lstlisting}
\end{wrapfigure}

A Listagem~\ref{lst:testeJasonIntSetup} mostra crenças que o agente terá
no inicio da simulação na plataforma \jason. Essas são populadas na memória
de acordo com a classe especificada na linha 8 da
Listagem~\ref{lst:testeJasonIntBase}. Esse processo já foi explicado na
seção~\ref{ch:p:ipjo}\todo{garantir isso}, a crença de \emph{step} como não
consta na ontologia será carregada na base de crenças padrão do \jason e as
demais mostradas serão inseridas na ontologia. Como pode ser observado, esse
processo é bastante transparente para o usuário.

Note também que o usuário precisou definir o limite para uma emoção virar
sentimento. No exemplo esse valor esta sendo definido como zero para que a
potência e a valência sejam iguais. Se for definido algum outro valor, a
potência será o valor total da emoção e esse valor não é de conhecimento do
agente. Ele conhece apenas a valência que será a diferença do valor total
menos o limite para ativação definido pela relação \emph{hasThreshold}. Por
exemplo, se o limite da pena (\emph{sorryFor}) é 6 e o valor atual é 8 então o
agente terá o sentimento com o valor 2 que será expresso por uma crença da
seguinte forma ``feeling("sorryFor",2).''.

%%%% fim da explicacao da aplicação interativa %%%%%

\begin{center}
    \begin{minipage}{140mm}
	\lstset{linewidth=140mm}
	\lstinputlisting[frame=trbl, caption=Arquivo de projeto do \jason para a
aplicação não interativa de teste,
label=lst:testeJasonNIBase]{../../sampletSummary/eoaus.mas2j}
    \end{minipage}
\end{center}

A aplicação não interativa utiliza a Listagem~\ref{lst:testeJasonNIBase} para
configurar o seu projeto. O ambiente especificado na linha 4 possui os mesmos
parâmetros do anterior com o acréscimo de um novo que indica uma ontologia.
Essa ontologia serve para o ambiente conhecer as rotinas dos agentes na
simulação, como deve ser desenhado o mapa exibido e as posições iniciais dos
agentes. O agente millie aqui é o utilizado para os testes e os limiares de
ativação de emoção estão configurados para valores diferentes. Existe uma
variedade de testes para as propriedades de objeto ou de dados e para as
instâncias de classes, além de testes das conclusões esperadas pela ontologia.
Veja o anexo~X\todo{botar o xodigo} para ver o código do agente.

\section{Simulando a Vida Cotidiana} \label{ch:cdu:svc}
\todo{era ``Simulando a Vida Cotidiana'' sem nada escrito}

...



%\chapter{Avaliação}

quais criterios?
porque desses criterios?
como foi feita a avaliacao dos criterios?

os resultados obtidos?
como se interpreta esses resultados com base nos criterios?


\chapter{Conclusão} \label{ch:cf}

%%Tem que essa frase ir aqui...
%Todo o código pode ser consultado no repositório disponível via
%\emph{GitHub}\footnote{http://github.com/rlucca/Maro}.

O presente trabalho apresentou um sistema que estende a base de
crenças do agente \jason para permitir que o mesmo tenha emoções segundo o
modelo desenvolvido por \citet{ortony1988cse}. O desenvolvimento se baseou em
utilizar uma ontologia para concluir qual seria a emoção sendo sentida. Dessa
forma, a base de crenças da plataforma foi alterada para manipular ontologias
permitindo, de maneira transparente, realizar inserções, remoções e consultas.

O desenvolvimento é uma ferramenta que ainda pode melhorar.
Primeiro, ao inserir crenças podem ser inseridas como parâmetro
átomos que são convertidos automaticamente para \emph{strings}. Todavia, esse
mesmo comportamento não se encontra expresso corretamente nas consultas e
remoções. Segundo, a empatia desenvolvida poderia ser informada pelo usuário
como um número. Por exemplo, zero são agentes distintos e um são agentes que
se consideram semelhantes por alguma razão. Terceiro, a relação de
probabilidade poderia ser números ao invés de conceitos.

A configuração para uma emoção ser disparada é trabalhosa. O
ideal seria ter um mecanismo abstrato utilizando as preferências de alguma
forma para concluírem as crenças que hoje precisam ser inseridas pelo usuário.
Dessa forma, a ideia de emoções serem transparente ao agente não foi
alcançada por completo. Todavia, se for necessário ser realizado um plano que só fica
ativo quando o agente possui um determinado grau de emoção então este teste
pode ser facilmente criado na área de contexto.

%Cada agente possui preferências diferentes. Essas preferências nos exemplos
%vistos não foram aproveitadas. Trabalhos futuros utilizando as preferências do
%agente para concluir as relações de avaliação necessárias para se disparar a
%emoção são interessantes porque podem gerar contribuições em mais de uma área
%e gera uma facilidade muito grande para os novos usuários da ferramenta.

As ontologias que foram desenvolvidas foram pensadas para não serem utilizadas
somente pela nossa ferramenta. Elas podem ser usadas em separado ou em conjunto.
Além disso, os exemplos construídos possuíram um tempo médio de passo de
simulação em torno de 2 segundos em média. Sendo que o primeiro passo foi sempre o
que teve um tempo maior de duração (em média 30 segundos), possivelmente, por
causa que é nele que é feita a leitura da ontologia e carga na base de crenças
iniciais que estão em código \emph{AgentSpeak}.



%\nocite{*} % Citar todos do arquivo bibtex, mesmo quando nao citado
%\bibliographystyle{abnt-alf} % eh o defauult do abntcite
\bibliography{mestrado}

%\singlespacing
%\appendix
%
\lstset{language=Java}
\lstset{linewidth=150mm}
\lstset{backgroundcolor=}
\lstset{breaklines=true}
\lstset{numbers=none}
\lstset{nolol=true} % no put it on table of content

%\chapter{Fontes da biblioteca desenvolvida}
%\subsection*{Arquivo do pacote \emph{maro.wrapper} - BBAffective.java}
%\lstinputlisting[]{appendix/BBAffective.java}
%
%\subsection*{Arquivo do pacote \emph{maro.wrapper} - Dumper.java}
%\lstinputlisting[]{appendix/Dumper.java}
%
%\subsection*{Arquivo do pacote \emph{maro.wrapper} - OwlApi.java}
%\lstinputlisting[]{appendix/OwlApi.java}
%%%%%%%%%%%%%%%%%%%%%%%%%%%%%%%%%%%%%%%%%%%%%%%%%%%%%%%%%%%%%%%%%%%%%%%%%%%%%%%
%\subsection*{ Arquivo do pacote \emph{maro.core} - ActionLoader.java}
%\lstinputlisting[]{appendix/ActionLoader.java}
%
%\subsection*{ Arquivo do pacote \emph{maro.core} - AnnotatedEnvironment.java}
%\lstinputlisting[]{appendix/AnnotatedEnvironment.java}
%
%\subsection*{ Arquivo do pacote \emph{maro.core} - BBKeeper.java}
%\lstinputlisting[]{appendix/BBKeeper.java}
%
%\subsection*{ Arquivo do pacote \emph{maro.core} - Emotion.java}
%\lstinputlisting[]{appendix/Emotion.java}
%
%\subsection*{ Arquivo do pacote \emph{maro.core} - EmotionKnowledge.java}
%\lstinputlisting[]{appendix/EmotionKnowledge.java}
%
%\subsection*{ Arquivo do pacote \emph{maro.core} - EnvironmentAction.java}
%\lstinputlisting[]{appendix/EnvironmentAction.java}
%
%\subsection*{ Arquivo do pacote \emph{maro.core} - FeelingsThreshold.java}
%\lstinputlisting[]{appendix/FeelingsThreshold.java}
%
%\subsection*{ Arquivo do pacote \emph{maro.core} - IntelligentEnvironment.java}
%\lstinputlisting[]{appendix/IntelligentEnvironment.java}
%%%%%%%%%%%%%%%%%%%%%%%%%%%%%%%%%%%%%%%%%%%%%%%%%%%%%%%%%%%%%%%%%%%%%%%%%%%%%%%
\chapter{Fontes da aplicação de teste não interativa}
\subsection*{Arquivo do projeto da plataforma - eoaus.mas2j} \label{atni}
\lstinputlisting[]{appendix/asl/sampletSummary/eoaus.mas2j}

\subsection*{Arquivo do agente Millie - millie.asl}
\lstinputlisting[]{appendix/asl/sampletSummary/millie.asl}

\subsection*{Arquivo dos demais agentes - nope.asl}
\lstinputlisting[]{appendix/asl/sampletSummary/nope.asl}

%%%%%%%%%%%%%%%%%%%%%%%%%%%%%%%%%%%%%%%%%%%%%%%%%%%%%%%%%%%%%%%%%%%%%%%%%%%%%%
\chapter{Fontes da aplicação de teste interativa}
\subsection*{Arquivo do projeto da plataforma - eoaus.mas2j}
\lstinputlisting[]{appendix/asl/sampletConsole/eoaus.mas2j}

\subsection*{Arquivo do agente Millie - millie.asl}
\lstinputlisting[]{appendix/asl/sampletConsole/millie.asl}

\subsection*{Arquivo de controle da interface interativa - AddGui.java}
\lstinputlisting[]{appendix/sampletConsole/AddGui.java}

%\subsection*{Arquivo da interface interativa - CharacterInspectorView.java}
%\lstinputlisting[]{appendix/sampletConsole/CharacterInspectorView.java}

\subsection*{Arquivo do ambiente - Environment.java}
\lstinputlisting[]{appendix/sampletConsole/Environment.java}
%%%%%%%%%%%%%%%%%%%%%%%%%%%%%%%%%%%%%%%%%%%%%%%%%%%%%%%%%%%%%%%%%%%%%%%%%%%%%%
\chapter{Fontes da aplicação de Futebol}

\subsection*{Arquivo de Projeto - soccer.mas2j} \label{adps}
\lstinputlisting[]{appendix/asl/sampletSoccer/soccer.mas2j}

\subsection*{Arquivo de ambiente - Environment.java}
\lstinputlisting[]{appendix/sampletSoccer/Environment.java}

%\subsection*{Arquivo de Ação - OkAction.java}
%\lstinputlisting[]{appendix/sampletSoccer/OkAction.java}
%
%\subsection*{Arquivo de Ação - addPerception.java}
%\lstinputlisting[]{appendix/sampletSoccer/addPerception.java}

\subsection*{Arquivo do Agente Stadium - stadium.asl}
\lstinputlisting[]{appendix/asl/sampletSoccer/stadium.asl}

%\subsection*{Arquivo \emph{AgentSpeak} - names.asl}
%\lstinputlisting[]{appendix/asl/sampletSoccer/names.asl}
%
\subsection*{Arquivo do Agente Watch1 - watch1.asl}
\lstinputlisting[]{appendix/asl/sampletSoccer/watch1.asl}

\subsection*{Arquivo do Agente Watch2 - watch2.asl}
\lstinputlisting[]{appendix/asl/sampletSoccer/watch2.asl}

\subsection*{Arquivo \emph{AgentSpeak} - emotionsWatch1.asl}
\lstinputlisting[]{appendix/asl/sampletSoccer/emotionsWatch1.asl}

\subsection*{Arquivo \emph{AgentSpeak} - emotionsWatch2.asl}
\lstinputlisting[]{appendix/asl/sampletSoccer/emotionsWatch2.asl}

\subsection*{Arquivo \emph{AgentSpeak} - common.asl}
\lstinputlisting[]{appendix/asl/sampletSoccer/common.asl}

\subsection*{Arquivo \emph{AgentSpeak} - appraisal.asl}
\lstinputlisting[]{appendix/asl/sampletSoccer/appraisal.asl}


%%%%%%%%%%%%%%%%%%%%%%%%%%%%%%%%%%%%%%%%%%%%%%%%%%%%%%%%%%%%%%%%%%%%%%%%%%%%%%
\chapter{Fontes da aplicação de Casa Virtual}

\subsection*{Arquivo de Projeto - sims.mas2j}
\lstinputlisting[]{appendix/sims.mas2j}

%\subsection*{Arquivo \emph{Java} - CharacterInspectorController.java}
%\lstinputlisting[]{appendix/CharacterInspectorController.java}
%
%\subsection*{Arquivo \emph{Java} - CharacterInspectorView.java}
%\lstinputlisting[]{appendix/CharacterInspectorView.java}

\subsection*{Arquivo \emph{Java} - House.java}
\lstinputlisting[]{appendix/House.java}

\subsection*{Arquivo \emph{Java} - HouseController.java}
\lstinputlisting[]{appendix/HouseController.java}

\subsection*{Arquivo \emph{Java} - HouseModel.java}
\lstinputlisting[]{appendix/HouseModel.java}

\subsection*{Arquivo \emph{Java} - HouseView.java}
\lstinputlisting[]{appendix/HouseView.java}

\subsection*{Arquivo \emph{Java} - AStar.java}
\lstinputlisting[]{appendix/AStar.java}

%\subsection*{Arquivo de Ação Interna - getItems.java}
%\lstinputlisting[]{appendix/getItems.java}

%\subsection*{Arquivo de Ação Interna - getItemsAtPlace.java}
%\lstinputlisting[]{appendix/getItemsAtPlace.java}

%\subsection*{Arquivo de Ação Interna - getPlaces.java}
%\lstinputlisting[]{appendix/getPlaces.java}

%\subsection*{Arquivo de Ação Interna - getPlacesByItem.java}
%\lstinputlisting[]{appendix/getPlacesByItem.java}

%\subsection*{Arquivo de Ação Interna - planRoute.java}
%\lstinputlisting[]{appendix/planRoute.java}

\subsection*{Arquivo de Ação - ChangeOrientationAction.java}
\lstinputlisting[]{appendix/ChangeOrientationAction.java}

\subsection*{Arquivo de Ação - ForwardAction.java}
\lstinputlisting[]{appendix/ForwardAction.java}

\subsection*{Arquivo de Ação - HideMeAction.java}
\lstinputlisting[]{appendix/HideMeAction.java}

\subsection*{Arquivo de Ação - NopeAction.java}
\lstinputlisting[]{appendix/NopeAction.java}

\subsection*{Arquivo de Ação - TryUseObjectAction.java}
\lstinputlisting[]{appendix/TryUseObjectAction.java}

\subsection*{Arquivo \emph{AgentSpeak} - appraisal.asl}
\lstinputlisting[]{appendix/asl/samplet/appraisal.asl}

\subsection*{Arquivo Comum dos Agentes - basicAgent3.asl}
\lstinputlisting[]{appendix/asl/samplet/basicAgent3.asl}

%\subsection*{Arquivo \emph{AgentSpeak} - routes.asl}
%\lstinputlisting[]{appendix/asl/samplet/routes.asl}
%
\subsection*{Arquivo \emph{AgentSpeak} - supportDiscoverLocation.asl}
\lstinputlisting[]{appendix/asl/samplet/supportDiscoverLocation.asl}

\subsection*{Arquivo \emph{AgentSpeak} - util.asl}
\lstinputlisting[]{appendix/asl/samplet/util.asl}


%%%%%%%%%%%%%%%%%%%%%%%%%%%%%%%%%%%%%%%%%%%%%%%%%%%%%%%%%%%%%%%%%%%%%%%%%%%%%%
\chapter{Ontologias}

\subsection*{Ontologia de preferências - annotation.owl}
\lstinputlisting[language=XML,inputencoding=latin1]{appendix/annotation.owl}

\subsection*{Ontologia de afetividade - occ-tbox.owl}
\lstinputlisting[language=XML,inputencoding=latin1]{appendix/occ-tbox.owl}

%\subsection*{Arquivo - annotation\_occ.owl}
%\lstinputlisting[language=XML,inputencoding=latin1]{appendix/annotation_occ.owl}
%
%\subsection*{Arquivo - UEM\_occ.owl}
%\lstinputlisting[language=XML,inputencoding=latin1]{appendix/UEM_occ.owl}
%
%\subsection*{Arquivo - sims.owl}
%\lstinputlisting[language=XML,inputencoding=latin1]{appendix/sims.owl}



\end{document}
