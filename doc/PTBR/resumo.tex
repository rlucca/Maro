% resumo na língua do documento
\begin{abstract}
Este trabalho apresenta um framework que permite a programação de agentes
capazes de perceberem seus próprios estados emocionais. O framework foi
desenvolvido em Java com base na plataforma multi-agente \emph{Jason}, estendendo a
base de crenças de agentes \emph{Jason} a fim de utilizar a ontologia afetiva
desenvolvida. Além disso, o ambiente foi construído a partir de uma base de
conhecimento que descreve rotinas em ambientes simulados. Um mecanismo de
avaliação das emoções baseando-se nas anotações dos objetos foi construído
apoiado por uma ontologia de preferência sobre essas anotações.
Dessa forma, aplicações de entretenimento poderiam utilizar o
sistema ou as bases de conhecimento apresentadas para diferentes propósitos. A criação de
um mapa onde os personagens atuam, e a criação da rotina de cada personagem e
suas preferências são alguns exemplos de utilizações. Para validação do
\emph{framework} desenvolvido, dois exemplos foram construídos. O primeiro
utilizou a maior parte dos grupos afetivos da ontologia proposta, com a
finalidade principal de demonstrar o modelo implementado. Já o segundo usa
apenas um grupo emotivo e serve para demonstrar a utilização conjunta de todas
as ontologias apresentadas.
\end{abstract}

%Example:
%http://www.sbgames.org/sbgames2011/proceedings/sbgames/papers/comp/full/11-92076_2.pdf
%1º, o que é o trabalho
%2º, o que uso e para que
%3º, para que posso utiliza-la
%4º, foi falado que um experimento para validar a arquitetura foi criado

% resumo na outra língua
% como parametros devem ser passados o titulo e as palavras-chave
% na outra língua, separadas por vírgulas
\begin{englishabstract}{Maro: A model of emotions using ontology}{virtual agents,
agent-oriented programming, simulation, affective computing, OCC's model, ontology}
This work presents a framework built on top of the Jason platform
\cite{bordini-jason} to allow the development of software agents that have
emotional states. The framework was developed in Java and extends the belief
base of Jason agents so as to use an ontology for the OCC affective model
\cite{ortony1988cse} that has been created as part of this work. The developed
belief base allows an agent to perceive its own emotions throw inferring new
beliefs based on the agent's appraisal of the state of the environment. In
addition, a model of agents' routine tasks was defined, as was a model for
agents' preferences about aspects of environment, helping automate the
ascription of emotional states. Finally, in order to validate the developed
framework, two applications were developed. The first demonstrates the use of
various different emotions from the affective model and the second uses in a
single application all the ontologies and models developed as part of this
work.
\end{englishabstract}
