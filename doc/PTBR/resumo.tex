% resumo na língua do documento
\begin{abstract}
Este trabalho apresenta um \emph{framework} que permite aos agentes perceberem
suas próprias emoções. Ele foi feito em \emph{Java} e usa a plataforma
multi-agente \jason \cite{bordini-jason}, sendo que estende a base de crenças
da última a fim de utilizar uma ontologia desenvolvida baseada no modelo
afetivo proposto por \citet{ortony1988cse}.
%
Além disso, uma ontologia de rotinas e de preferência sobre as anotações
foram criadas para serem utilizadas no ambiente. Dessa forma, aplicações de
entretenimento poderiam utilizar as três ontologias juntas para criar um mapa,
para criar as rotinas de cada agente e para dar preferências sobre as
anotações utilizadas nas percepções vindas do ambiente. Para validação desse
desenvolvimento, um exemplo de jogo de futebol foi utilizado e este atingiu os
principais grupos emotivos do modelo afetivo. Além disso, um outro exemplo
usando as três ontologias foi criado com a finalidade de validação do sistema.
\end{abstract}

%Example:
%http://www.sbgames.org/sbgames2011/proceedings/sbgames/papers/comp/full/11-92076_2.pdf
%1º, o que é o trabalho
%2º, o que uso e para que
%3º, para que posso utiliza-la
%4º, foi falado que um experimento para validar a arquitetura foi criado

% resumo na outra língua
% como parametros devem ser passados o titulo e as palavras-chave
% na outra língua, separadas por vírgulas
%\begin{englishabstract}{Maro: An emotional model using ontology}{virtual
%agents, programming, agents, simulation, affective computing, OCC's model, ontology}
%This work presents a framework built to work with Jason's platform
%\cite{bordini-jason} to allow agents acknowledge their own emotions. This work have
%been developed in \emph{Java} and extended the belief base from agent of
%platform to utilize an developed ontology based on the affective model
%\cite{ortony1988cse}. For so the development of a belief base permits an agent
%perceive emotions based on your appraisal of environment and conclude new
%beliefs. In addition, an ontology to make the schedules of agents and to make
%yours preferences about annotations on belief were developed. So,
%entertainment applications and games could use the three ontologies together
%to create a map of the city, to create schedules or routines of each
%agent and to give preferences about annotations used in beliefs that came from
%environment. Finally, to validate the framework has been developed an
%soccer's application to demonstrate many emotions from affective model.
%\end{englishabstract}
