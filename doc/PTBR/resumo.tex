% resumo na língua do documento
\begin{abstract}
Este trabalho apresenta um framework que permite a programação de agentes
capazes de perceberem seus próprios estados emocionais. O framework foi
desenvolvido em Java com base na plataforma multi-agente \emph{Jason}, estendendo a
base de crenças de agentes \emph{Jason} a fim de utilizar a ontologia afetiva
desenvolvida. Além disso, o ambiente foi construído a partir de uma base de
conhecimento que descreve rotinas em ambientes simulados. Um mecanismo de
avaliação das emoções baseando-se nas anotações dos objetos foi construído
apoiado por uma ontologia de preferência sobre essas anotações.
Dessa forma, aplicações de entretenimento poderiam utilizar o
sistema ou as bases de conhecimento apresentadas para diferentes propósitos. A criação de
um mapa onde os personagens atuam, e a criação da rotina de cada personagem e
suas preferências são alguns exemplos de utilizações. Para validação do
\emph{framework} desenvolvido, dois exemplos foram construídos. O primeiro
utilizou a maior parte dos grupos afetivos da ontologia proposta, com a
finalidade principal de demonstrar o modelo implementado. Já o segundo usa
apenas um grupo emotivo e serve para demonstrar a utilização conjunta de todas
as ontologias apresentadas.
\end{abstract}

%Example:
%http://www.sbgames.org/sbgames2011/proceedings/sbgames/papers/comp/full/11-92076_2.pdf
%1º, o que é o trabalho
%2º, o que uso e para que
%3º, para que posso utiliza-la
%4º, foi falado que um experimento para validar a arquitetura foi criado

% resumo na outra língua
% como parametros devem ser passados o titulo e as palavras-chave
% na outra língua, separadas por vírgulas
%\begin{englishabstract}{Maro: An emotional model using ontology}{virtual
%agents, programming, agents, simulation, affective computing, OCC's model, ontology}
%This work presents a framework built to work with Jason's platform
%\cite{bordini-jason} to allow agents acknowledge their own emotions. This work have
%been developed in \emph{Java} and extended the belief base from agent of
%platform to utilize an developed ontology based on the affective model
%\cite{ortony1988cse}. For so the development of a belief base permits an agent
%perceive emotions based on your appraisal of environment and conclude new
%beliefs. In addition, an ontology to make the schedules of agents and to make
%yours preferences about annotations on belief were developed. So,
%entertainment applications and games could use the three ontologies together
%to create a map of the city, to create schedules or routines of each
%agent and to give preferences about annotations used in beliefs that came from
%environment. Finally, to validate the framework has been developed an
%soccer's application to demonstrate many emotions from affective model.
%\end{englishabstract}
