% resumo na língua do documento
\begin{abstract}
A área de computação afetiva é a área de estudo das emoções dentro da
computação.  O presente trabalho foca na sub-área de síntese de emoções, com o
objetivo de simular o comportamento afetivo entre atores.  Essa ferramenta,
baseada na plataforma \jason \cite{bordini-jason}, utiliza uma ontologia
baseada no modelo de emoções de \citet{ortony1988cse}.  Além disso, conceitos
vistos em diferentes áreas como animação comportamental \cite{bates1994role} e
ambientes inteligentes \cite{doyle1998annotated} são utilizados neste
trabalho.
\end{abstract}

% resumo na outra língua
% como parametros devem ser passados o titulo e as palavras-chave
% na outra língua, separadas por vírgulas
%\begin{englishabstract}{}{Electronic document preparation, \LaTeX, ABNT, UFRGS\todo{olhar depois} }
%This document is an example on how to prepare documents at II/UFRGS
%using the \LaTeX\ classes provided by the UTUG\@. At the same time, it
%may serve as a guide for general-purpose commands. \emph{The text in
%the abstract should not contain more than 500~words.}
%\end{englishabstract}
