% resumo na língua do documento
\begin{abstract}
O estudo das emoções é feito dentro da computação na área chamada computação
afetiva. Essa área estuda tudo relacionado ou surgido das emoções e o presente
trabalho foca na síntese de emoções utilizando o modelo baseado em
significados proposto por \citet{ortony1988cse} na plataforma \jason
\cite{bordini-jason}. Assim, o desenvolvimento de uma integração entre \jason
e ontologias foi feito. As emoções são guardadas em uma ontologia desenvolvida
e se uma crença for concluída como emoção, ela vira um sentimento quando o
valor atingir um determinado limite mínimo. Dessa forma, a emoção (não
perceptível) passa a ser um sentimento percebido.
\end{abstract}

% resumo na outra língua
% como parametros devem ser passados o titulo e as palavras-chave
% na outra língua, separadas por vírgulas
\begin{englishabstract}{Maro: An emotional model using ontology}{virtual agents, programming of
agents, simulation, affective computing, OCC's model, ontology}
The study of emotions is made in computing at the field called affective
computing. That field studies all things related or emerged from emotions. The
present work is focused on emotional synthesis using a meaning-based emotion
model \cite{ortony1988cse} at \jason platform \cite{bordini-jason}. Thus, the
development of a integration between \jason and ontology was created. All
emotions are saved in an ontology development. Therefore, the ontology is used
to classify the belief of an agent as emotion and when that emotion hits a
threshold it becomes a feeling. In that way, the emotion (not perceived)
passes to be a perceived feeling.
\end{englishabstract}

