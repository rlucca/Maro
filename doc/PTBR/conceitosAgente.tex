\subsection{Visões Diferentes sobre o Conceito Agente}

\citet{laird2001human} afirmam que a pesquisa em Inteligência Artificial tem
sido fragmentada em muitas áreas especializadas e, assim, algoritmos
específicos e mais eficientes podem ser criados.
%
Há várias definições do conceito de Agente, por exemplo
\cite{shoham1993agent,roadmap,fatemeh2009multi}. \citet{shoham1993agent}
propõe que Agente é uma entidade definida por componentes mentais como
crenças, habilidades, escolhas e comprometimentos.

\citet{franklin1997agent} define um agente através das características que a
entidade deve apresentar.  Atualmente, o menor conjunto comumente aceito
dessas características é composto por três conceitos principais: (i)
Autonomia; (ii) Sociabilidade; (iii) Situacionalidade.
\citet{roadmap,fatemeh2009multi} concordam que autonomia é quando um Agente
toma suas próprias decisões independente de qualquer outra entidade do sistema
ou, ainda, da intervenção diretas de seres humanos. A característica de
sociabilidade permite flexibilidade na execução das tarefas, através da
interação com outros agentes que estejam presentes no sistema. A última,
permite que o agente situe-se em um ambiente dinâmico interagindo com o mesmo
através de algum tipo de sensor ou atuador.

\citet{ingrand1992architecture} indicam que entidades autônomas podem ter a
capacidade de expor seus dados internos para que seja possível um usuário,
possivelmente humano, dar dicas sobre a forma de resolução dos problemas sendo
enfrentados. Esta visão não conflita com a noção de autonomia exposta acima,
pois o agente permanece independente para tomar suas decisões podendo rejeitar
as dicas ou sugestões enviadas pelo usuário.

\citeauthor{ingrand1992architecture} também define Agente em um ambiente com
componentes heterogêneos e com diferentes tempos de resposta para a execução
de suas tarefas. \citet{doyle1998annotated} estendem o conceito dizendo que os
objetos pertencentes ao ambiente devem conter anotações que determinam a forma
de uso dos objetos disponibilizados neles. Assim, não se sabe como todos os
objetos funcionam e, sim, uma forma de aprender com o próprio objeto a sua
forma de utilização. \citet{shoham1993agent} define sociabilidade como uma
habilidade cognitiva necessária para o desempenho das suas tarefas.

Essas inúmeras definições do termo Agente não permite saber se ele é um ser
físico ou abstrato. Dessa forma, \citet{nareyek2001review,damiano2008emotions}
defendem que o agente é o ser abstrato de um ator físico. Em outras palavras,
o ser que age ou atua no meio é chamado de personagem ou ator. Enquanto a
mente desse ser é chamada de agente.

