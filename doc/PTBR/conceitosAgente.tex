\subsection{Visões Diferentes sobre o Conceito Agente} \label{sec:aopca}

\citet{laird2001human} afirmam que a pesquisa em Inteligência Artificial tem
sido fragmentada em muitas áreas especializadas e, assim, algoritmos
específicos e mais eficientes podem ser criados. Um exemplo disso é as
inúmeras definições do conceito Agente. \citet{shoham1993agent}
propõe que Agente é uma entidade definida por componentes mentais como
crenças, habilidades, escolhas e comprometimentos.

\citet{franklin1997agent} define um Agente através das características que a
entidade deve apresentar.  Atualmente, o menor conjunto comumente aceito
dessas características é composto por três conceitos principais: (i)
Autonomia; (ii) Sociabilidade; (iii) Situacionalidade.
\citet{roadmap,fatemeh2009multi} concordam que Autonomia é quando um Agente
toma suas próprias decisões independente de qualquer outra entidade do sistema
ou, ainda, da intervenção direta de seres humanos. A característica de
Sociabilidade permite flexibilidade na execução das tarefas, através da
interação com outros Agentes que estejam presentes no sistema. Por exemplo, é
possível negociar que outro Agente faça a tarefa. A última característica,
permite que o agente situe-se em um ambiente dinâmico interagindo com o mesmo
através de sensores e atuadores.

\citet{ingrand1992architecture} indicam que entidades autônomas podem ter a
capacidade de expor seus dados internos para que seja possível um usuário,
possivelmente humano, dar dicas sobre a forma de resolução dos problemas sendo
enfrentados. Esta visão não conflita com a noção de Autonomia exposta acima,
pois o agente permanece independente para tomar suas decisões podendo rejeitar
as sugestões.

\citeauthoronline{ingrand1992architecture} também definem Agente em um ambiente com
componentes heterogêneos e com diferentes tempos de resposta para a execução
de suas tarefas. \citet{doyle1998annotated} estendem o conceito dizendo que os
objetos no ambiente devem conter anotações que determinam a sua utilização
pelos Agentes. Assim, não se sabe como todos os objetos funcionam e, sim, uma
forma de aprender com o mesmo a sua forma de utilização.
\citet{shoham1993agent} define Sociabilidade como uma habilidade cognitiva
necessária para o desempenho das suas tarefas.

Essas inúmeras definições do termo Agente não permitem saber se ele é um ser
físico ou abstrato. Dessa forma, \citet{nareyek2001review,damiano2008emotions}
defendem que o Agente é o ser abstrato de um Ator físico. Em outras palavras,
o ser que age ou atua no meio é chamado de Personagem ou Ator. Enquanto, a
mente desse ser é chamada de Agente. Essa diferenciação é utilizado no
presente, como forma de desambiguação.
