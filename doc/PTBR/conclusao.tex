\chapter{Conclusão} \label{ch:cf}

O presente trabalho apresentou um \emph{framework} que estende a base de
crenças do agente \jason para permitir que o mesmo tenha emoções segundo o
modelo desenvolvido por \citet{ortony1988cse}. O desenvolvimento se baseou em
utilizar uma ontologia para concluir qual seria a emoção sendo sentida. Dessa
forma, a base de crenças da plataforma foi alterada para manipular ontologias
permitindo, de maneira transparente, realizar inserções, remoções e consultas.

O \emph{framework} desenvolvido é uma ferramenta que ainda pode melhorar.
Primeiro, ao utilizar inserir crenças podem ser inseridas como parâmetro
átomos que são convertidos automaticamente para \emph{strings}. Todavia, esse
mesmo comportamento não se encontra expresso corretamente nas consultas e
remoções. Segundo, a empatia desenvolvida poderia ser informada pelo usuário
como um número. Por exemplo, zero são agentes distintos e um são agentes que
se consideram semelhantes por alguma razão. Terceiro, cada agente para ter sua
avaliação classificada corretamente precisa incluir como suas crenças valores
equivalentes as relações esperadas.

Assim sendo, a configuração para uma emoção ser disparada é trabalhosa. O
ideal seria ter um mecanismo abstrato utilizando as preferências de alguma
forma para concluírem as crenças que hoje precisam ser inseridas pelo usuário.
Dessa forma, a ideia de emoções serem transparente ao agente não foi
alcançada. Todavia, se for necessário ser realizado um plano que só fica
ativo quando o agente possui um determinado grau de emoção então este teste
pode ser facilmente criado na área de contexto.

%Cada agente possui preferências diferentes. Essas preferências nos exemplos
%vistos não foram aproveitadas. Trabalhos futuros utilizando as preferências do
%agente para concluir as relações de avaliação necessárias para se disparar a
%emoção são interessantes porque podem gerar contribuições em mais de uma área
%e gera uma facilidade muito grande para os novos usuários da ferramenta.

As ontologias que foram desenvolvidas foram pensadas para não serem utilizadas
só pela nossa ferramenta. Elas podem ser usadas em separado ou em conjunto. Na
ontologia afetiva, o principal ponto a melhorar é deixar a probabilidade ser
um valor numérico e não conceitos que representam faixas de valores.
%
Além disso, os exemplos construídos possuíram um tempo médio de passo de
simulação em torno de 1 à 2 segundos em média. O primeiro passo foi sempre o
que teve um tempo maior de duração, possivelmente, por causa que é nele que é
feita a leitura da ontologia e carga na base de crenças do que esta em
arquivos de código \jason como crenças iniciais.


