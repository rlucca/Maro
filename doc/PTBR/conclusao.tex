\chapter{Conclusão} \label{ch:cf}

O presente trabalho apresentou um mecanismo desenvolvido para a plataforma
\jason que permite aos agentes da simulação possuírem sentimentos. Todo o
código do sistema, incluído os exemplos e o presente texto, pode ser acessado
pelo repositório \emph{on-line} hospedado via
\emph{GitHub}\footnote{http://github.com/rlucca/Maro}.
Os apêndices presentes visam facilitar a consulta de alguns dos artefatos lá
presentes e debatidos aqui.
O presente capítulo esta dividido em duas seções. A primeira serve para
avaliar o sistema desenvolvido e discutir melhorias futuras. A segunda seção,
serve para apresentar as considerações finais fazendo um pequeno retrospecto
sobre o que se esperava do trabalho inicialmente e o que foi alcançado.


\section{Avaliação e Trabalhos Futuros}

Para avaliar o sistema criado, os critérios levados em consideração são de
simplicidade de uso e configuração. Uma coisa é considerada simples de usar
quando requer nenhum esforço para ser usada, enquanto simplicidade de
configuração quer dizer que se o usuário desejar alterar determinados
parâmetros ele deve ser capaz disso sem um grande esforço. Alguns exemplos de
parâmetros podem ser as próprias emoções porque o usuário deseja utilizar
apenas um subconjunto delas ou ser a alteração do processo de avaliação que
resulta no valor da emoção.

Conforme dito anteriormente no capítulo~\ref{ch:cdu}, a configuração mínima
exigida é que o usuário especifique na configuração do \jason a base de
crenças desenvolvida. Fora isso, o sistema ainda precisa conhecer os limites
mínimos para uma emoção virar sentimento que pode ser informada de duas
formas. A primeira é como crenças iniciais dos agentes, conforme demostrado na
Listagem~\ref{lst:testeJasonIntSetup}~(pág.~\pageref{lst:testeJasonIntSetup}),
porém essa forma de uso é extremamente cansativa porque se precisa informar
várias coisas. Cada uma das emoções precisa ser configurada, ao todo existem
22 delas e para cada uma são necessárias 3 relações que dizem de quem é a
configuração, de qual emoção esta sendo configurada e de qual é o valor mínimo
para sua ativação. O segundo é o formato utilizado pela
seção~\ref{ch:cdu:home}~(pág.~\pageref{ch:cdu:home}). Esse armazena esses
valores de configuração na própria ontologia. Sendo assim, basta informar para
a base de crenças a ontologia correta e essa configuração não precisa ser
feita via código \emph{AgentSpeak}.

Agora, a plataforma \jason será capaz de criar as percepções das emoções
quando for adicionada uma percepção ``step''. Sendo assim, o ambiente ou algum
agente deve informar essa crença para os demais.  Essas duas formas foram
usadas na seção~\ref{ch:cdu:tbc}~(pág.~\pageref{ch:cdu:tbc}). Entretanto, cabe
salientar novamente que antes de o agente acreditar que essa crença foi
recebida ele deveria limpar as crenças existentes sobre seus sentimentos para
evitar duplicações (ver figura da pág.~\pageref{fig:testeJasonIntBase}).

Logo, a necessidade de uma percepção ``step'' ser adicionada à base de crenças
sempre que se deseja fazer uma avaliação das emoções atuais afeta o projeto do
usuário. Essa é uma necessidade do mecanismo construído invasiva porque torna
o sistema do usuário descontinuo nessa parte das emoções. Além disso, ele deve
se preocupar com quem irá informar essa percepção. Eliminar essa necessidade é
um plano futuro pois torna o desenvolvido mais transparente permitindo-o ser
melhor utilizado. Essa atualização pode ser colocada no momento da consulta da
percepção ``feeling'' ou em algum outro ponto.

Hoje se o usuário desejar utilizar apenas um subconjunto de emoções isso não é
possível. Ele sempre precisa configurar todos os limites mínimos de todas as
emoções e, no decorrer da simulação, não inserir as referidas relações para
aquelas emoções se tornarem ativas. Isso no futuro será ajustado para o
sistema aguardar somente as emoções presentes no conceitos \emph{Emotion} e
que exista um modelo de percepção baseado nas preferências do agente para
tornar a escrita de código menor.
Além disso, remover a necessidade de o cálculo da potência da emoção ser feita
em \emph{Java} para ser interessante. A vantagem de fazer isso em código
\emph{AgentSpeak} é uma maior transparência no que esta sendo feito e como se
esta fazendo quando for necessário fugir da forma proposta. Entretanto,
alterar o cálculo hoje é possível estendendo a classe \emph{OwlApi} de leitura
da ontologia com a finalidade de sobrecarregar as funções \emph{summaryOf} que
realizam esses cálculos.

De maneira similar, as consultas de crenças feitas em código \emph{AgentSpeak}
são transparentes para o usuário quanto ao local de armazenamento. Entretanto,
se o usuário precisar saber onde a crença esta sendo salva. Ele pode verificar
a existência da anotação ``ontology''. Se ela existir então a crença esta
sendo armazenada na ontologia especificada. Caso contrário, ela esta sendo
guardada na base de crenças da plataforma.

Infelizmente, o mecanismo de consulta de crenças não esta ainda perfeito e
exige que o usuário quando faça uma consulta utilize sempre \emph{strings} ou
números inteiros. Além disso, como foi visto nos exemplos do
capítulo~\ref{ch:cdu}, a responsabilidade de ligar as percepções as relações
necessárias à ontologia afetiva é do usuário. Mesmo sendo capaz de fazer uso
da ontologia de preferência, todas as emoções não podem ainda ser concluídas
só com base no que se encontra configurado. Isso se deve principalmente ao
fato de que algumas emoções são complexas exigindo conhecimento de que um
personagem esta interferindo com um determinado objetivo ou que uma meta afeta
outra ou que uma ação é fora dos padrões do ator.

Sobre a ontologia afetiva, ela foi testada de diferentes formas na integração
com o \emph{Jason}.
Os pontos de melhora dessa
ontologia são três. O primeiro seria a inclusão do conceito de empatia
(pág.~\pageref{mark:empat}) como sendo um número a ser informado pelo usuário.
O segundo é a remoção do conhecimento de amigo e inimigo. Esse conhecimento
pode ou deveria ser deduzido a partir das emoções \emph{Love} e \emph{Hate}.
Terceiro, transformar o conceito de probabilidade em uma relação numérica.
Essas alterações não melhoram a facilidade do usuário, mas aumentam o controle
sobre o ator.
%Além disso, a última tornaria possível testar técnicas de aprendizado juntamente com a afetividade.

\section{Considerações Finais}

Um sistema que estende a base de crenças do agente \jason para permitir que o
mesmo tenha emoções segundo o modelo desenvolvido por \citet{ortony1988cse}
foi apresentado. Assim, o desenvolvimento se baseou em utilizar uma ontologia
para classificar qual é a emoção sendo sentida com base nas crenças e
percepções do agente.
Dessa forma, a base de crenças da plataforma foi alterada para manipular
ontologias permitindo, de maneira transparente, realizar inserções, remoções e
consultas.

Inicialmente, o objetivo do trabalho era simular a construção do comportamento
afetivo usando os agentes da plataforma \jason de maneira integrada com alguma
aplicação em outra linguagem. Todavia, criar um outro mecanismo para possibilitar o
\jason rodar externamente e adaptar a aplicação especifica para utilizá-lo foi
considerado fora do escopo do trabalho. Sendo por isso, removido da lista de
objetivos almejados.

O objetivo principal foi a construção de uma ferramenta que permita definir o
comportamento emocional de personagens virtuais. Sendo por isso, as ontologias
seus principais elementos por serem artefatos reutilizáveis. Infelizmente, o
objetivo de ter uma configuração mínima não foi alcançado e nem o objetivo de
não possuir regras. Entretanto, esse último objetivo foi cancelado
porque na versão 2 da \emph{OWL} as regras são parte da linguagem
e, por isso, foram utilizadas.

Os principais resultados são as aplicações construídas com o sistema. Elas
servem, principalmente, para validar o sistema desenvolvido e possuem
diferentes propósitos conforme foi visto no capítulo anterior. Ao todo foram
feitas quatro aplicações e demonstram que mesmo sem um conceito de decaimento
automático do valor da emoção o mecanismo desenvolvido funciona de uma maneira
adequada.

