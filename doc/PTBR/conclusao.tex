\chapter{Considerações finais}

O presente artigo visa apresentar uma visão do trabalho que está sendo feito
para a construção de um simulador no qual os atores e ambientes interagem.
Entretanto, a ontologia foi omitida porque está ainda sendo aprimorada.  A
idéia apresentada não é nova, porém, o ponto chave é a ontologia permitir aos
agentes raciocinarem sobre suas emoções de forma transparente.  Essa
transparência é útil porque permite ao animador conhecer e, até mesmo,
especificar o personagem com um grau bastante elevado de abstração.
%
Dessa forma, dois agentes com a mesma especificação, ao enfrentarem a mesma
situação, terão o mesmo comportamento quando tiverem vivenciado as mesmas
experiências. Essa experiência inclui as informações adquiridas sobre o
ambiente e sobre os demais atores e não somente os interesses e metas do
mesmo.

O modelo OCC em uso no presente trabalho, foi definido em
\citeyear{ortony1988cse} e possui ao todo 22 emoções especificadas por regras
que demonstram quando a mesma acontece. Entretanto, em nenhum momento da sua
definição foi tratado explicitamente como uma emoção afeta as outras presentes
no personagem.
%
A ferramenta em desenvolvimento tem como interesse a simulação computacional
do comportamento humano e, dessa forma, ela pode ser utilizada de diferentes
maneiras e pode coletar diferentes informações para análises com diversos
propósitos.
