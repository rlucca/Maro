\chapter{Conclusão} \label{ch:cf}

O presente trabalho apresentou um \emph{framework} que estende a base de
crenças do agente \jason para permitir que o mesmo tenha emoções segundo o
modelo desenvolvido por \cite{ortony1988cse}. O desenvolvimento se baseou em
utilizar uma ontologia para concluir qual seria a emoção sendo sentida. Dessa
forma, a base de crenças manipula a ontologia inserindo, removendo e
consultando-a conforme necessário e de maneira transparente.

Todavia, a ideia de permitir aos agentes raciocinarem sobre suas emoções de
maneira transparente não foi alcança. Cada agente tem uma preferência e cada
um precisa acrescentar as relações para disparar aquela emoção da maneira que
se deseja. Tentar reduzir esse problema parece um trabalho futuro interessante
porque abre portas para uma facilidade imensa.

As ontologias em todo caso são um trabalho a parte. Elas podem ser usadas em
separado ou em conjunto, no \emph{framework} ou em outra aplicação. O
principal ponto a melhorar na ontologia digo que é transformar a probabilidade
em um valor numérico e não conceitos que representam faixas de probabilidade.

Além disso, a interface construída foi considerada rápida. Cada passo da
simulação ficou em torno de 1 à 2 segundos em média, com exceção do primeiro
passo no qual o tempo fica perto de 30 segundos por causa que é feita a
leitura do arquivo e carga na base de crenças do que esta em arquivos de
código \jason como crenças iniciais.


