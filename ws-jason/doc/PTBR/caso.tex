\chapter{Exemplo de Uso} \label{chap-casoWS}

O presente capítulo apresenta dois exemplos desenvolvidos. Na
seção \ref{casoWS-room} se apresenta o exemplo \emph{room} adaptado para
executar dois agentes (\emph{porter} e \emph{paranoid}) na linguagem Python.
Já o seguinte apresenta o exemplo
\emph{game of life}, adaptado para ter o ambiente desenvolvido em Haskell. Os
agentes em Python estão na seção \ref{casoWS-gol}.

\section{Exemplo \emph{Room}} \label{casoWS-room}

O exemplo \emph{Room} foi adaptado para executar dois agentes de maneira
externa à plataforma Jason. Assim, a Listagem~\ref{lst-roomMas2j} (pg.
\pageref{lst-roomMas2j}) deve ficar conforme a Listagem~\ref{lst-roomXR}.
Fora isso, os dois programas que implementam o servidor de agentes do
\emph{porter} e do \emph{paranoid} devem ser disparados antes da plataforma
Jason iniciar a simulação. A implementação desses pode ser vista
no Anexo~\ref{anexo-room}.

\lstset{linewidth=100mm}
\begin{center}
    \begin{minipage}{100mm}
	\begin{lstlisting}[frame=trbl, caption=Arquivo de projeto do Jason adaptado., label=lst-roomXR]
// Isso eh um comentario
MAS room {
  infrastructure: Centralised
  environment: RoomEnv
  executionControl: jason.control.ExecutionControl
  agents:
    porter
    [url="http://localhost:8080"]
    agentArchClass maro.architecture.AgArchXmlRpc
    ;
    claustrophobe
    ;
    paranoid
    [url="http://localhost:8081"]
    agentArchClass maro.architecture.AgArchXmlRpc
    ;
}
	\end{lstlisting}
    \end{minipage}
\end{center}

A Listagem~\ref{lst-roomXR} define uma simulação na plataforma Jason com três
agentes heterogêneos. O primeiro chamado \emph{porter} encontra-se localizado
em um WSS definido pela url ``http://localhost:8080'' que define o
servidor sendo a máquina local na porta 8080. Nessa url poderia ser colocado
lugares diferentes sem nenhum problema e a porta poderia ser omitida quando a
mesma for a 80. O segundo agente nomeado \emph{claustrophobe} é um agente
Jason e o último agente é outro WSS localizado na porta 8081.

A implementação dos agentes em Jason podem ser consultadas na
Listagem~\ref{lst-agente} (página~\pageref{lst-agente}). O agente \emph{porter}
como pode ser visto, deve trancar a porta quando receber do agente
\emph{paranoid} a meta \emph{locked} e deve destrancar a porta quando receber
do agente \emph{claustrophobe} a meta $\sim$\emph{locked}. De acordo com a
Figura~\ref{fig-uml-ag} (pg \pageref{fig-uml-ag}), há 7 métodos que o servidor
necessita implementar. Os métodos \emph{handshake}, \emph{login} e
\emph{logout} não serão comentados aqui.

\lstset{linewidth=130mm}
\begin{center}
    \begin{minipage}{140mm}
	\begin{lstlisting}[frame=trbl, caption=Método de percepção e ação do agente \emph{porter}., label=lst-roomWS-perceive]
def perceive(id, percepts):
	data = agentL[id]
	agentL[id] = (data[0], percepts, data[2])
	return act(id, False)

def act(id, eraseAll):
	if eraseAll:
		data = agentL[id]
		agentL[id] = (data[0], [], data[2])
	data = agentL[id]
	if len(data[2]) >= 1:
		ret = exclusion_act(data[1], data[2][0])
		if len(ret) > 0:
			agentL[id] = (data[0], data[1], data[2][1:])
		return ret
	return ""

def exclusion_act(perceptions, message_received):
	M = { "~locked(door)":"locked(door)", "locked(door)":"~locked(door)" }
	P = { "~locked":"lock", "locked":"unlock" }
	test = M[ message_received[4] ][:-1].split('(')
	for p in perceptions:
		functor = p[0]
		terms = p[1]
		annots = p[2]
		if functor == test[0] and terms[0] == test[1]:
			return P[functor]
	return ""
	\end{lstlisting}
    \end{minipage}
\end{center}

Na Listagem~\ref{lst-roomWS-perceive} tem-se a implementação de dois dos
métodos essenciais. O primeiro (\emph{perceive}) recebe o identificador do
agente criado na inicialização e a nova lista de percepções e usa esses
dados para atualizar a tupla que contêm, respectivamente, o nome do agente, as
percepções e as mensagens recebidas. Agora, o método \emph{act} deve retornar
a ação que o agente \emph{porter} fará no momento e, para isso, recebe o
identificador do agente e um valor lógico que indica se as percepções devem
ser apagadas ou não. Assim, é possível chamar esse método sem passar por
\emph{perceive} quando o grupo de novas percepções for o grupo vazio. Portanto, a
primeira atividade dessa função é verificar esse valor e limpar as percepções
caso seja necessário. Após, o uso verifica se há mensagens, havendo chama
uma função auxiliar para definir se há ou não ação a realizar e consome a
mensagem.

\lstset{linewidth=130mm}
\begin{center}
    \begin{minipage}{140mm}
	\begin{lstlisting}[frame=trbl, caption=Métodos relacionados com a comunicação do agente \emph{porter}., label=lst-roomWS-messages]
def receiveMessages(id, messages):
	data = agentL[id]
	msgL = []
	for msg in messages:
		m = msg[1:-1].split(',')
		if m[1] == "claustrophobe" and m[4] == "~locked(door)":
			msgL.append(m);
		elif m[1] == "paranoid" and m[4] == "locked(door)":
			msgL.append(m);
	agentL[id] = (data[0], data[1], msgL)
	return sendMessages(id)

def sendMessages(id):
	return []
	\end{lstlisting}
    \end{minipage}
\end{center}

O agente que origina à mensagem é testado no momento do recebimento
da mensagem, onde o agente pode descarta-la ou guarda-la.
Esse comportamento pode ser observado na Listagem~\ref{lst-roomWS-messages}.
Nela os métodos \emph{checkMail} e \emph{getMessages} mapeiam,
respectivamente, para \emph{receiveMessages} e \emph{sendMessages}. O método
de recebimento de mensagens, \emph{receiveMessages}, sabe como parsear a tupla
que veio (linha 5) e somente cadastra as mensagens relevantes para desempenhar
as mesmas ações do código Jason. Dessa forma, ele só considera as mensagens
relevantes que para o agente \emph{paranoid} é um pedido de trancamento
e para o agente \emph{claustrophobe} é um pedido de destrancamento da porta.
O agente \emph{porter} não envia nenhum tipo de mensagens para
outros agentes, assim a implementação do \emph{getMessages} mapeado para
\emph{sendMessages} na implentação retorna um conjunto vazio representando que
não deseja enviar mensagens.

\lstset{linewidth=130mm}
\begin{center}
    \begin{minipage}{140mm}
	\begin{lstlisting}[frame=trbl, caption=Métodos do agente \emph{paranoid}., label=lst-roomWS-paranoid]
def act2(id, eraseAll):
	if eraseAll:
		data = agentL[id]
		agentL[id] = (data[0], [], data[2])
	data = agentL[id]
	if len(data[1]) > 0:
		sendMessage = exclusion_act(data[1])
		if len(sendMessage) > 0:
			data[2].append(sendMessage)
			agentL[id] = (data[0], data[1], data[2])
	return ""

def exclusion_act(perceptions):
	p = perceptions[0]
	functor = p[0]
	terms = p[1]
	annots = p[2]
	if functor == "~locked" and terms[0] == "door":
		try:
			msgId = agentL["msgId"]
		except KeyError:
			msgId = 0
		agentL["msgId"] = msgId + 1
		return "<mprpc%02d,,achieve,porter,locked(door)>" % msgId
	return ""

def sendMessages(id):
	data = agentL[id]
	ret = []
	if len(data[2]) > 0:
		ret = data[2]
		agentL[id] = (data[0], data[1], [])
	return ret
	\end{lstlisting}
    \end{minipage}
\end{center}

No agente \emph{paranoid} a situação se altera um pouco, a função de percepção
(\emph{perceive}), ação (\emph{act}) e recebimento de mensagens
(\emph{checkMail}) não são mostrados na Listagem~\ref{lst-roomWS-paranoid}.
O método de percepção é igual ao mostrado
anteriormente, porém ele passa a chamar \emph{act2} ao invés de \emph{act}.
Já no método \emph{act}, o método \emph{act2} só é chamado quando
precisa limpar todas as crenças. O método \emph{checkMail}
simplesmente realiza a chamada a função \emph{getMessages} (mapeada na
implementação para \emph{sendMessages}) para enviar para a plataforma Jason as
mensagens desejadas.

A presente implementação considera uma mensagem como uma string. Essa string
tem o seguinte formato: ``<id, emissor, tipo, vitima, mensagem>''. Esses dados
são usados para construir uma tupla com 5 elementos, conforme já explicado na
seção \ref{sec-WSC}. Essa string é construída na função interna
\emph{exclusion\_act}, chamada somente quando há percepções, e
guardada para ser enviada depois pela função \emph{getMessages}. Essa função
retorna as mensagens que o agente deseja enviar e limpa o campo de
mensagens à serem enviadas.


\section{Exemplo \emph{Game of Life}} \label{casoWS-gol}

O presente exemplo tem como foco apresentar uma implementação do
ambiente como WS. O serviço segue a interface apresentada na
Figura~\ref{fig-uml-env} presente na página~\pageref{fig-uml-env}, onde
existem 10 métodos.
%
O presente foi organizado em um diretório e um arquivo \emph{Python}. No arquivo
\emph{Python} tem-se a implementação do raciocínio dos agentes e no diretório
tem-se a implementação do ambiente. O ambiente é tanto a implementação da API
pública em WS quanto a interface que o usuário pode interagir (que na versão
Jason existia).

\lstset{linewidth=130mm}
\begin{center}
    \begin{minipage}{140mm}
	\begin{lstlisting}[frame=trbl, caption=Arquivo de projeto do \emph{Game-of-Life}., label=lst-gol]
MAS game_of_life {
  infrastructure: Centralised(pool,2)
  environment: maro.Environment.EnvXmlRpc("http://localhost:8079")
  executionControl: jason.control.ExecutionControl

  agents:
    cell
    [url="http://localhost:8080/RPC2"]
    agentArchClass maro.architecture.AgArchXmlRpc
    #3600 // matrix 60x60
    ;
}
	\end{lstlisting}
    \end{minipage}
\end{center}

A Listagem~\ref{lst-gol} mostra o arquivo de projeto adaptado. Na URL a parte
com ``RPC2'' poderia ser omitida deixando somente o endereço com a porta. O
ambiente recebe a URL da maneira como demostrada. A infraestrutura utilizada é
a centralizada, porém é utilizado um conjunto de \emph{threads} que no caso
foi definido no valor 2 (duas \emph{threads}).

\lstset{linewidth=73mm}
\begin{wrapfigure}{l}{83mm}
    \begin{minipage}{73mm}
	\begin{lstlisting}[frame=trbl, caption=Tipos usados pelo ambiente.,label=lst-gol-types]
type Action = Percept
data Percept = Percept {
        functor::String,
        params::[String],
        annots::[Percept]
    } deriving (Show,Eq,Ord)
data TPerception = Perceptions {
        global::[Percept],
        local::[(String, Percept)],
        updated::[String]
    } deriving (Show)
	\end{lstlisting}
    \end{minipage}
\end{wrapfigure}

Antes de mostrar os trechos da implementação da parte do ambiente em
\emph{Haskell} é necessário falar sobre os dois tipos de dados que são usados para
suportar o desenvolvimento. Na Listagem~\ref{lst-gol-types} tem-se os tipos, o
primeiro denominado \emph{Percept} é a tupla de percepções explicada na
seção~\ref{sec-WSS}. Esse tipo de dado também pode ser referenciado como
\emph{Action}. Já o tipo \emph{TPerception} contém, respectivamente,
uma lista de percepções, uma lista com tuplas que mapeam nome do agente para uma
percepção e uma lista de nomes de agentes. Esse tipo serve para guardar
separadamente as percepções globais dos locais e guarda ainda os agentes que
estão atualizados (o cliente já requisitou os dados atuais). Cabe salientar,
ainda, que todas as listas são mantidas ordenadas para facilitar a busca
nas mesmas.

Na Listagem~\ref{lst-gol-arcP} são mostrados os métodos implementados pelo WSS
para adicionar, remover e limpar as percepções locais ou globais.
Esses métodos tratam a lista de
percepções que eles alteram como conjuntos. Dessa forma, não é necessário se
preocupar com percepções repetidas porque elas não vão existir. Além disso,
nas operações de remoção e limpeza é utilizada a
diferença entre dois conjuntos. Essas funções da listagem encontram-se
mapeadas e publicadas por um WS, porém, quando comparadas com a
Tabela~\ref{table-server-environment} (pg \pageref{table-server-environment})
pensa-se que há parâmetros em excesso. Na implementação do método
\emph{addPerceptLocal}, por exemplo, têm-se por parâmetros uma referência ao
tipo interno \emph{TPerceptions}, uma \emph{string} (nome do agente), uma
tupla de percepção e como retorno um booleano encapsulado. Esse protótipo de
função é possível pois quando se realiza o mapeamento das funções publicadas
pelo WS passa-se alguns parâmetros da chamada a priori.
Essa característica é utilizada em quase todas as funções e,
principalmente, na \emph{performAction}. Consulte no Anexo~\ref{anexo-gol} a
função \emph{serving} (pg~\pageref{anexo-gol-endLuccaWS}) para maiores
detalhes.

\lstset{linewidth=140mm}
\begin{center}
    \begin{minipage}{140mm}
	\begin{lstlisting}[frame=trbl, caption=Métodos relativos às percepções.,label=lst-gol-arcP]
addPercept :: IORef(TPerception) -> Percept -> IO Bool
addPercept p perception = do
    (Perceptions gp lp ua) <- readIORef p
    let gp' = insertSet perception gp
    modifyIORef p $ \(Perceptions _ lp _) -> (Perceptions gp' lp [])
    return True

addPerceptLocal :: IORef(TPerception) -> String -> Percept -> IO Bool
addPerceptLocal p name perception = do
    (Perceptions gp lp ua) <- readIORef p
    let lp' = insertSet (name,perception) lp
    modifyIORef p $ \(Perceptions gp _ _) -> (Perceptions gp lp' [])
    return True

removePercept :: IORef(TPerception) -> Percept -> IO Bool
removePercept p perception = do
    (Perceptions gp lp ua) <- readIORef p
    let gp' = gp `minus` [perception]
    modifyIORef p $ \(Perceptions _ lp _) -> (Perceptions gp' lp [])
    return True

removePerceptLocal :: IORef(TPerception) -> String -> Percept -> IO Bool
removePerceptLocal p name perception = do
    (Perceptions gp lp ua) <- readIORef p
    let lp' = lp `minus` [(name,perception)]
    modifyIORef p $ \(Perceptions gp _ _) -> (Perceptions gp lp' [])
    return True

clearPercepts :: IORef(TPerception) -> IO Bool
clearPercepts p = do
    (Perceptions gp lp ua) <- readIORef p
    if (null gp) then
        return True
     else do
        modifyIORef p $ \(Perceptions _ lp _) -> (Perceptions [] lp [])
        return True

clearPerceptsLocal :: IORef(TPerception) -> String -> IO Bool
clearPerceptsLocal p name = do
    (Perceptions gp lp ua) <- readIORef p
    if (null gp) then
        return True
     else do
        let lp' = lp `minus` [(name, perception) | (n, perception) <- lp, n==name]
        modifyIORef p $ \(Perceptions _ lp _) -> (Perceptions [] lp [])
        return True
	\end{lstlisting}
    \end{minipage}
\end{center}

O ambiente é responsável por armazenar todas as percepções que os agentes
possuem, assim ele deve também disponibilizar formas desses dados serem
recuperados. A Listagem~\ref{lst-gol-testP} mostra os dois métodos que são
usados no momento do agente conhecer as suas percepções. A função
\emph{havePercepts} utiliza a lista de agentes atualizados para devolver
verdadeiro quando é necessário ocorrer uma chamada ao método
\emph{getPercepts}. O  método \emph{getPercepts} define as percepções à
serem retornadas na linha 9 da Listagem~\ref{lst-gol-testP} como sendo a
união das percepções globais com o conjunto de percepções locais filtrados
pelo nome do agente.

\begin{center}
    \begin{minipage}{140mm}
	\begin{lstlisting}[frame=trbl, caption=Métodos para recuperar as percepções., label=lst-gol-testP]
havePercepts :: IORef(TPerception) -> String -> IO Bool
havePercepts p name = do
    pp <- readIORef p
    return $ not $ (updated pp) `has` name

getPercepts :: IORef(TPerception) -> String -> Bool -> IO [Percept]
getPercepts p name isUpdate = do
    (Perceptions gp lp ua) <- readIORef p
    let r = union gp [lp' | (n, lp') <- lp, n == name]
    if isUpdate == False then
        return r
     else do
        let ua' = insertSet name ua
        modifyIORef p $ \(Perceptions gp lp _) -> (Perceptions gp lp ua')
        return r
	\end{lstlisting}
    \end{minipage}
\end{center}

A presente implementação do ambiente aceita quatro ações possíveis vindo do
agente Jason. As ações nomeadas são \emph{skip}, \emph{live} e \emph{die}, uma
quarta ação existe e é a ação de ignorar para quando o ambiente não conhece-la não
realizar nenhuma tarefa. Essa seleção é feita baseado no operador da ação
recebida (linha 5 da Listagem~\ref{lst-gol-action}).

\lstset{linewidth=130mm}
\begin{center}
    \begin{minipage}{140mm}
	\begin{lstlisting}[frame=trbl, caption=Método que realiza ações., label=lst-gol-action]
performAction :: IORef(TPerception) -> SyncModel -> GLWidget -> String
     -> Action -> IO Bool
performAction p controller viewer name action =
    let
        actionName = functor action
        actionPars = params action
        actionBy = name
    in
        case actionName of
            "skip"  -> cmdRealize p controller viewer actionBy Nothing
            "live"  -> cmdRealize p controller viewer actionBy (Just True)
            "die"   -> cmdRealize p controller viewer actionBy (Just False)
            _       -> return True -- ignore action
	\end{lstlisting}
    \end{minipage}
\end{center}

Na Listagem~\ref{lst-gol-action}, a chamada a função interna \emph{cmdRealize}
permitiu que fosse abstraído código. Essa é a função que
realiza o trabalho ``sujo'' de atualizar o modelo e solicitar a reexibição da
interface com o usuário. Os únicos parâmetros úteis para a célula são: a
referência as percepções, quem está realizando a ação e que tipo de ação é.
Para representar o tipo de ação é utilizado o tipo \emph{Maybe Bool}. Dessa
forma, as funções \emph{skip}, \emph{live} e \emph{die} são mapeadas,
respectivamente, para não ser um tipo booleano ou ser um tipo booleano valendo
verdadeiro ou ser um tipo booleano valendo falso. Maiores detalhes consulte o
Anexo~\ref{anexo-gol}.

