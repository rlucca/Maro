\chapter{Conclusão} \label{chap-conclusao}

No presente trabalho foi apresentado uma forma de aumentar a integração da
plataforma Jason com outras linguagens além do Java. A forma apresentada
utiliza o conceito de serviços para permitir a utilização de agentes e
ambientes heterogêneos, isto é, diferentes componentes da plataforma podem
rodar em ambientes computacionais diferentes.

Assim, ao se construir os métodos do servidor deve-se tomar cuidado para
evitar problemas. Um método quando implementado incorretamente pode refletir
somente em outro ponto. Dessa forma, é importante no futuro ter-se bibliotecas
específicas para cada linguagem visando dar uma implementação padrão correta à
todas as funções necessárias ao serviço web e seu desenvolvimento pode ser
conforme demanda.

O exemplo do \emph{Game Of Life} apresentado na seção \ref{casoWS-gol} possui
um ciclo médio de 28 segundos para receber a comunicação de todos os agentes.
Enquanto que, na plataforma para todos os 3600 agentes (matriz de 60 por 60) o
tempo é inferior a 1 segundo. Conforme já dito, o protocolo de comunicação
é o \emph{XML-RPC} baseado na linguagem \emph{XML}. Em
\cite{kohlhoff2003evaluating} é mostrado que a linguagem \emph{XML} pode
deixar as
mensagens até 10 vezes maiores e que o principal tempo gasto na comunicação
de um processo para outro é perdido no transformar o dado de binário para
\emph{XML} e vice-versa.

Dessa forma, a lentidão no sistema pode ser pensada como sendo culpa da
quantidade de agentes e da representação textual dos dados trocados. O ideal
seria trocar a representação textual e, também, rever o protocolo para estudar
se há uma forma de evitar a troca de mensagens desnecessárias. Por exemplo, na
implementação atual, se um agente não recebe e não envia mensagens para outros
agentes a plataforma Jason chamará a função \emph{checkMail} e obterá um
retorno vazio todas as vezes. Outro exemplo é o mapeamento de todas as ações
do ambiente em uma única função evitando que seja conhecido as ações
válidas através da consulta da listagem de métodos do servidor.

O mecanismo de segurança criado pelo protocolo desenvolvido visa dar um mínimo de
segurança aos servidores, porém é necessário pensar em mecanismos mais
fortes para evitar o problema do homem no meio. Talvez permitir uma
entrada no arquivo de configuração do projeto ativando uma cifragem à ser
utilizada nas mensagens trocadas. Além disso, o uso de serviços
permitiria a criação de uma ferramenta de teste que torna possível a
substituição de um dos componentes (cliente ou um dos servidores) para se
validar o outro componente. Esse teste seria realizado conhecendo as entradas e a
saída esperada para a verificação do comportamento esperado. Essa forma de
teste é conhecida por teste unitário e considera todo o serviço sendo testado
como uma só unidade.

