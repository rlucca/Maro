% resumo na língua do documento
\begin{abstract}
O presente trabalho visa aumentar a integração da plataforma Jason \cite{bordini-jason} com
programas não desenvolvidos em linguagem Java. A finalidade é permitir que outras linguagens
sejam utilizadas quando for vantajoso. Com essa finalidade, as principais abordagens em serviços Web (WS)
foram estudadas \cite{cerami2002web,allman2003evaluation,kohlhoff2003evaluating}. As duas
abordagens mais conhecidas são baseadas na tecnologia XML: XML-RPC e SOAP.
Logo, um serviço Web será desenvolvido para tornar o ambiente e os agentes da plataforma Jason
capazes de serem implementados em diferentes linguagens. Além disso, um levantamento teórico
da área será realizado explicando os protocolos mencionados e trabalhos relacionados na área
\cite{piunti2009soa,bellifemine1999jade}.
%O presente trabalho visa aumentar a integração da plataforma Jason \cite{bordini-jason} com
%programas não desenvolvidos em linguagem Java. Assim sendo, foi estudado
%as principais abordagens de integração: (i) Sockets; (ii) RPC; (iii) XML-RPC; (iv) SOAP.
%No entanto, como se está interessado em tecnologias de serviços Web (WS), as
%abordagens (i) e (ii) não são interessantes de serem utilizadas.
%As abordagens que sobram são relevantes para o uso em WS \cite{cerami2002web}.
%Ambas utilizam
%a linguagem XML para trocar informações, sendo o protocolo SOAP o mais recente.
%A escolha de qual abordagem utilizar fez uso do estudo realizado e de uma
%pesquisa prévia sobre as bibliotecas disponíveis nas seguintes linguagens:
%C, Haskell, Java, Lisp, Lua, Java, PHP e Python.
%
%Para o aumento da integração da plataforma Jason será desenvolvido um WS com duas
%finalidades. A primeira finalidade é permitir que tanto agentes quanto o ambiente
%sejam compartilhados com as demais aplicações. Para o ambiente, os itens compartilhados
%são objetos presentes e as ações disponíveis para se realizar.
%Já os agentes devem perceber as modificações feitas no ambiente, deliberar
%quanto ao que se deseja realizar utilizando para isso suas crenças e percepções para
%concluir uma ação no ambiente ou mental (interna no próprio agente). Entretanto,
%a segunda finalidade é prover ao programador a oportunidade de escolher se
%deseja implementar em outra linguagem o ambiente ou os agentes ou, ainda, ambos.
\end{abstract}

% resumo na outra língua
% como parametros devem ser passados o titulo e as palavras-chave
% na outra língua, separadas por vírgulas
%\begin{englishabstract}{}{Electronic document preparation, \LaTeX, ABNT, UFRGS\todo{olhar depois} }
%This document is an example on how to prepare documents at II/UFRGS
%using the \LaTeX\ classes provided by the UTUG\@. At the same time, it
%may serve as a guide for general-purpose commands. \emph{The text in
%the abstract should not contain more than 500~words.}
%\end{englishabstract}
