% introdução
\chapter{Introdução} \label{chap-introducao}

%% Resolvi comentar porque parece um adendo muito fraco, alem disso quem for avaliar o trabalho tem que ser da area entao falar mais uma vez isso eh ***
%Antes de introduzir a plataforma, os conceitos de agente e sistema multi-agente
%serão brevemente explicados. De acordo com \cite{roadmap}, o agente
%é um sistema de computador que se encontra inserido em um ambiente
%sendo capaz de interagir de maneira flexível e autônomo no meio. Da mesma
%forma, um sistema multi-agente é um sistema projetado e implementado para
%tirar proveito das interações entre esses agentes.
%\todo{fico em duvida se devo continuar e falar dos outros que consideram Agente = Personagem + sistema especialista e Objeto = Personagem... Mais ou menos seria a ideia do cartago}

Uma das arquiteturas mais conhecidas para o desenvolvimento de agentes
cognitivos é a arquitetura BDI, onde
\cite{dastani2009modularity,de2005planning} são dois trabalhos que utilizam-na.
A plataforma Jason \cite{bordini-jason}, baseada na arquitetura BDI,
 utiliza uma extensão da linguagem de programação abstrata \cite{rao1996agentspeak},
 chamada AgentSpeak(L), para especificar as decisões
dos agentes. Enquanto, os ambientes são unicamente códigos Java que, por
muitas, vezes demostram graficamente os agentes atuando.

No processo de deliberação, as ações encontram-se agrupadas em um conjunto
chamado de plano e cada plano possui um conjunto de ações para
verificar o contexto de sua execução \cite{dastani2009modularity}.
O contexto serve para determinar os planos válidos para executar no presente
momento. Em caso de haver mais de um plano válido, o primeiro será escolhido.

Uma decisão de um agente é a escolha feita para se realizar
determinada atividade. Por exemplo: o agente X deseja empurrar o agente Y.
Essa atividade ou ação de empurrar pode ser implementada de diferentes formas. Na
primeira, o agente X encaminha uma mensagem para o agente Y dizendo que ele
esta sendo empurrado. Na segunda, o agente X envia uma mensagem para o
ambiente dizendo que esta empurrando o outro agente.
Ambas implementações são possíveis em Jason e utilizam diferentes
mecanismos. O que deve ficar claro aqui é a importância das decisões de
projeto e, também, de se manter a coesão em projetos de MAS.

A motivação desse trabalho é permitir que a plataforma Jason se integre com
outras linguagens que podem não ser do mesmo paradigma. Essa integração seria
possível usando a interface de funções nativas do Java, porém essa solução
deixaria nosso código com um nível de acoplamento elevado e restrito a um
conjunto limitado de linguagens.
%\todo{essas duas frases estao muito fortes}
%Essa integração seria possível usando a interface de funções nativas do Java
%utilizadas quando uma biblioteca é feita em C ou C++. Todavia, essa solução
%deixaria atrelado a plataforma Jason somente a uma linguagem e o nível de
%acoplamento aumentaria.
Assim sendo, o uso de WS surge naturalmente para solucionar esses dois
problemas. As principais abordagens usadas em WS utilizam XML-RPC e SOAP
\cite{cerami2002web}. Padrões mais novos existem, porém não com bibliotecas
prontas para o conjunto de linguagens escolhidas ou não foi considerado um
padrão tão conhecido.

Para o trabalho
desenvolvido atingir o maior número de linguagens possíveis, um levantamento em
linguagens imperativas (C e Java), funcionais (Haskell e Lisp) e de script
(Lua, PHP e Python) foi realizado. Todas essas linguagens possuem ou suporte
nativo ou suporte através de bibliotecas que podem ser instaladas para ambos
os protocolos. Por exemplo: PHP tem suporte nativo aos dois protocolos; Java
tem suporte nativo ao SOAP; Python tem suporte nativo ao XML-RPC. Dessa forma,
ambos os protocolos foram considerados equivalentes ficando a decisão de qual
utilizar baseando-se no detalhamento dos mesmos.

No capítulo \ref{chap-levantamentoBibliografico},
o estudo detalhado dos protocolos, a plataforma Jason e abordagens que
utilizam o conceito de serviços web para integrar sistemas são discutidas.
No capítulo \ref{chap-desenvolvimentoWS}, os critérios utilizados na decisão e
o projeto de implementação podem ser encontrados. No capítulo \ref{chap-casoWS}
é mostrado um passo-a-passo da montagem do ambiente e duas implementações que
utilizam o desenvolvimento. Por fim, no capítulo \ref{chap-conclusao} é
debatido a experiência e os trabalhos futuros.


% http://www.tldp.org/HOWTO/XML-RPC-HOWTO/xmlrpc-howto-competition.html#xmlrpc-howto-soap
% http://weblog.masukomi.org/writings/xml-rpc_vs_soap.htm
% descontar uma palavra da nota de rodape
% C = lib
% Haskell = lib
% Lisp = lib
% Lua = lib
% Java = xml-rpc via lib e SOAP nativo
% PHP = XML-RPC e SOAP nativo
% Python = XML-RPC nativo

