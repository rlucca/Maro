\chapter{Introdução}

%Ontologia foi caracterizada como o estudo da existência, desde Aristóteles,
%por estar interessada em descrever todas as coisas existentes e as relações
%que existem entre elas. Atualmente, essa forma de representar o domínio do
%conhecimento tem se tornado popular na computação por causa da independência
%de sistema oferecida.

Segundo \citet{gruber1993translation}, uma ontologia é uma especificação
explícita do domínio sendo tratado. Além disso, o resultado é um conjunto de
termos, fundamentados e bem formados para se trabalhar. Entretanto
\citet{ontoly2004Approach}, definiu ontologia como uma representação usada
para descobrir outras informações no domínio. Atualmente, ontologias são
vistas como um entendimento comum e compartilhado de um domínio que pode ser
utilizado na comunicação entre máquinas ou entre pessoas
\cite{wks2008towards}.

%% O paragrafo acima foi baseado nesse...
%Segundo \citet{gruber1993translation}, uma ontologia é uma especificação
%explícita do domínio sendo tratado. Além disso, ele aplica esse conceito em
%sistemas baseados em conhecimento no qual o domínio é representado por um
%formalismo declarativo e o conjunto de conceitos e relações formam o
%vocabulário usado pelo sistema. Esse vocabulário provém um conjunto de termos
%bem formados para se trabalhar.
%%
%Entretanto \citet{ontoly2004Approach}, definiu ontologia como uma representação
%do conhecimento usado para descobrir outras informações ou conhecimentos sobre
%o assunto. Atualmente, ontologias são vistas como um entendimento comum e
%compartilhado de um domínio que pode ser utilizado na comunicação entre
%máquinas ou entre pessoas \cite{wks2008towards}.

O presente trabalho foi baseado na linguagem \OWL. Essa linguagem foi
regulamentada pela \emph{W3C}\footnote{Ver
\url{http://www.w3.org/standards/semanticweb/ontology}.}, orgão internacional
que regulamenta padrões na Web, para ser usada na \emph{Web} Semântica. Essa
linguagem foi criada em 2002 com o proposito de criação de ontologias e
trabalha com a hipótese de mundo aberto, isto é, nada é afirmado por não ser
dito. Infelizmente, para a \emph{W3C} não há uma distinção clara entre vocabulário e
ontologia.

A linguagem \OWL permite a especificação de conceitos e de suas instâncias,
porém não é possível descrever uma regra simples como um conceito de igualdade
no qual duas relações distintas tem que chegar na mesma instância final.
%Outro exemplo, o conceito de tios que são os irmãos de meus pais não é
%possível ser feita na \OWL. A versão 2 da \OWL permite a descrição do conceito
%de tios, porém o conceito de igualdade permanece impossível.
%
Entretanto, a versão 2 da \OWL passou a incorporar um mecanismo de regras.
Dessa forma, regras lógicas podem ser usadas para melhorar a precisão dos
conceitos sendo descritos porque permite lidar com as suas instâncias. Assim,
é possível descrever o exemplo anterior.

%% Esse era o paragrafo anterior ali reescrito
%A linguagem \emph{SWRL}\footnote{Mais detalhes \url{http://www.w3.org/Submission/SWRL/}.}
%recomendada pela \emph{W3C} permite escrever regras lógicas que melhoram a
%precisão dos conceitos sendo descritos porque permite lidar com as suas
%instâncias. Dessa forma, a \emph{SWRL} supri uma falta até então não tratada pela
%linguagem \OWL e, por isso, seu uso em conjunto é extremamente poderoso. Essas
%duas linguagens juntas permitem a escrita do conceito de igualdade descrito
%anteriormente. A versão 2 da \OWL tornou essa junção desnecessária porque a
%mesma incorporou o mecanismo de regras.

A área da computação que estuda as emoções foi denominada Computação Afetiva por
\citet{Pic98}. As emoções, segundo \citet{damasio2004erro}, podem ser divididas
entre primárias (não-cognitivas) e secundárias (cognitivas). As emoções
primárias surgem a partir de reações a determinados estímulos externos e são
geradas rapidamente. Já as emoções secundárias são aprendidas ao longo da
nossa vida, isto é, são geradas por uma avaliação de uma situação de acordo
com nossos objetivos e valores morais. Entretanto, essa divisão ainda não esta
consolidada porque o fato de haver menos atividade cognitiva não quer dizer
que esta atividade não exista.

O presente trabalho pretende estudar diferentes ontologias do modelo afetivo
visando o entendimento destas e suas diferenças no capítulo~\ref{cap:eda}.
Entretanto, nenhum trabalho propôs a junção de uma ontologia afetiva com uma
de humanos virtuais com uma que explique como tratar as percepções do
ambiente. Esse assunto é tratado no capítulo~\ref{cap:tp} juntamente com um
exemplo de como se usa o desenvolvimento. O capítulo~\ref{cap:c} apresenta as
conclusões e trabalhos futuros.
