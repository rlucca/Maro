\chapter{Introdução}

Segundo \citet{gruber1993translation}, uma ontologia é uma especificação
explícita do domínio sendo tratado. Além disso, o resultado é um conjunto de
termos, fundamentados e bem formados para se trabalhar. Entretanto
\citet{ontoly2004Approach}, definiu ontologia como uma representação usada
para descobrir outras informações no domínio. Atualmente, ontologias são
vistas como um entendimento comum e compartilhado de um domínio que pode ser
utilizado na comunicação entre máquinas ou pessoas \cite{wks2008towards}.

O presente trabalho foi baseado na linguagem \OWL. Essa linguagem foi
regulamentada pela \emph{W3C}\footnote{Ver
\url{http://www.w3.org/standards/semanticweb/ontology}.}, orgão internacional
que regulamenta padrões na Web, para ser usada na \emph{Web} Semântica. Essa
linguagem foi criada em 2002 com o proposito de criação de ontologias e
trabalha com a hipótese de mundo aberto, isto é, nada é afirmado por não ser
dito. Infelizmente, para a \emph{W3C} não há uma distinção clara entre vocabulário e
ontologia.

A linguagem \OWL permite a especificação de conceitos e de suas instâncias,
porém não é possível descrever uma regra simples como um conceito de igualdade
no qual duas relações distintas tem que chegar na mesma instância final.
%Outro exemplo, o conceito de tios que são os irmãos de meus pais não é
%possível ser feita na \OWL.
%
Entretanto, a versão 2 da \OWL passou a incorporar um mecanismo de regras.
Dessa forma, regras lógicas podem ser usadas para melhorar a precisão dos
conceitos sendo descritos porque permite lidar com as suas instâncias. Assim,
é possível descrever o exemplo anterior.

A área da computação que estuda as emoções foi denominada Computação Afetiva por
\citet{Pic98}. As emoções, segundo \citet{damasio2004erro}, podem ser divididas
entre primárias (não-cognitivas) e secundárias (cognitivas). As emoções
primárias surgem a partir de reações a determinados estímulos do ambiente e são
geradas rapidamente. Já as emoções secundárias são aprendidas ao longo da
nossa vida, isto é, são geradas por uma avaliação de uma situação de acordo
com nossos objetivos e valores morais. Entretanto, essa divisão ainda não esta
consolidada porque o fato de haver menos atividade cognitiva não quer dizer
que esta atividade não exista.

O presente trabalho pretende desenvolver uma ontologia de modelo afetivo que
integre as emoções, preferências e a rotina dos personagens em um mundo
virtual. No capítulo~\ref{cap:eda} diferentes ontologias do modelo afetivo, de
humano virtual ou do comportamento do mesmo são discutidas. O trabalho
proposto juntamente com uma seção de demonstração é relatado no
capítulo~\ref{cap:tp}. No \ref{cap:c}, as conclusões e algumas direções
futuras são comentadas.  Os fontes das ontologias desenvolvidas podem ser
vistos no anexo~\ref{cap:fo}.

