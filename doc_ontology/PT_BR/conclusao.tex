\chapter{Conclusão} \label{cap:c}
%2345678901234567890123456789012345678901234567890123456789012345678901234567890

A questão inicial desse projeto foi a modelagem do modelo proposto por
\citet{ortony1988cse} de maneira a ser o mais próximo possível do completo. A
restrição inicial de não utilizar regras para o desenvolvimento da mesma teve
que ser removida por causa que o \OWL foca no nível dos conceitos e quando se
quer dizer algo mais explícito como, por exemplo, o agente avaliando e o
agente avaliado são diferentes é uma tarefa impossível. Entretanto, as regras
operam focando no nível dos indivíduos e, assim, o problema foi resolvido.

Algumas limitações na ontologia desenvolvida existem quando comparadas com o
modelo original. Um dos casos mais fortes é a falta da empatia que permitiria
uma pessoa se sentir orgulhosa ou envergonhada por outra pessoa. Por exemplo,
se ter vergonha de quem bate nos filhos ou ter orgulho por seu time ter ganho.
Algumas simplificações menores foram que os conceitos \emph{Joy} e
\emph{Distress} passaram a se relacionar com o domínio de probabilidade para
deixar claro que eles não tem probabilidade. Além disso, \emph{Love} e
\emph{Hate} possui duas relações que por ser interessante só quando ambas
relações eram positivas ou negativas foi usado só uma relação.

A ferramenta \emph{Protege}\footnote{Mais informações, consulte \url{http://protege.stanford.edu}.}
em sua versão 4.1 foi a utilizada para o desenvolvimento das ontologias.
Primeiro, por ser uma ferramenta bastante conhecida. Segundo, por que ele
suportar a linguagem \OWL 2 que foi a utilizada no presente trabalho.
%
Cabe salientar que a ontologia de anotação passou por diferentes fases. Por
exemplo, em alguns momentos ela foi chamada de ambiente e em outros de
percepção. Entretanto, no final o nome terminou sendo de anotação. A principal
ideia dessa ontologia de ``percepção'' (que agora é chamada de anotação) era
permitir o usuário conhecer somente ela sem conhecer as demais. Todavia, isso
violaria a modularização das mesmas e, dessa forma, foi escolhido remover essa
asserção.

A ontologia final é complexa porque a mesma realiza processamento enquadrando
os indivíduos em seus respectivos conceitos. Os testes foram feitos na
ferramenta de desenvolvimento da ontologia visando testar a criação de todas
as emoções em uma ontologia separada. Assim, os resultados foram validados
sempre depois de alguma modificação com o raciocinador \emph{Hermit}.

%Isso aqui é bom de falar na dissertação...
%A performance em testes simples como os mostrados no capitulo X\todo{fazer?},
%tendem a uma execução que pode levar de 2 ate 5 segundos que para aplicações
%interativas seria muito ineficiente.

