\chapter{Conclusão} \label{cap:c}
%2345678901234567890123456789012345678901234567890123456789012345678901234567890

\todo{se a secao 3.2 ficar na 16 e a 3.4 na pagina 17 ou 18 entao remover
referencia a pagina da secao 3.4 na 16}
% Esse paragrafo tem q ser encaixado ou aqui ou na introducao!
A ferramenta \emph{Protege}\footnote{Mais informações, consulte \url{http://protege.stanford.edu}.}
em sua versão 4.1 foi a utilizada para o desenvolvimento das ontologias.
Primeiro, por ser um dos mais conhecidos. Segundo, por que ele suporta a
linguagem \OWL 2.
%

\todo{escrever denovo}A questão inicial desse projeto era como modelar o modelo OCC
\cite{ortony1988cse} de maneira a ser o mais próximo possível do completo. As
restrições iniciais de não utilizar regras propostas no inicio do mesmo caíram
por terra ao se ter problemas porque a linguagem de descrição escolhida tem
foco principal nos conceitos e não nos indivíduos. Todavia, ao utilizar regras
o que era necessario fazer nos individuos foi completamente sanado.

Algumas limitacoes do modelo aqui feito, foi que ele nao possui a força de
unidade cognitva (empatia) por considerar que esta causaria maiores problemas
de conceitualização. Além disso, no modelo orignal as emoções de \emph{Joy} e
\emph{Distress} não possuem probabilidade aqui. No entanto, elas se relacionam
com o conceito de nenhuma probabilidade.

A performance em testes simples como os mostrados no capitulo X\todo{fazer?},
tendem a uma execução que pode levar de 2 ate 5 segundos que para aplicações
interativas seria muito ineficiente.

...

conclusao,
resultados (revisa-los)
e trabalhos futuros possiveis


%quando concluir tem que ver se deu problema de hifenizacao em algum lugar!
