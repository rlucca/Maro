\documentclass[12pt]{sa}
\usepackage{geometry}			% ver geometry.pdf para outros leaiutes
\geometry{a4paper}
\usepackage[T1]{fontenc}        % pacote para conj. de caracteres correto
\usepackage[utf8]{inputenc}	    % pacote para acentuação
\usepackage[alf]{abntcite}		% pacote para as referencias da abnt nao numerica
\usepackage{graphicx}           % pacote para importar figuras
\usepackage{times}              % pacote para usar fonte Adobe Times
\usepackage{color}              % pacote para usar \textcolor
\usepackage{rotating}		   	% pacote para rotacoes
\usepackage{xcolor,colortbl}	% colorir tabelas
\usepackage{url}

% See the ``Article customise'' template for come common customisations

\title{Provisorio: Estudo sobre ontologias afetivas}
\author{Ricardo Rodrigues Lucca}
%\date{}
\newcommand{\citet}{\citeonline}

%%% BEGIN DOCUMENT
\begin{document}

\maketitle


%\section{Introdução}
%
%Segundo \citet{wks2008towards}, ontologia foi caracterizada como o estudo da
%existência, desde Aristóteles, por estar interessada em descrever todas as
%coisas existentes e as relações que existem entre elas. Atualmente, essa forma
%de representar o domínio do conhecimento tem se tornado popular na computação
%por causa da independência de linguagem oferecida.
%
%O conceito de ontologia utilizado aqui é o dado por
%\citet{ontoly2004Approach}. Eles definiram este como sendo uma representação
%de conhecimento utilizada para capturar conhecimento e informação sobre
%determinado assunto e que é, normalmente, estruturado de forma similar a uma
%rede semântica. Dessa forma, uma ontologia pode ser pensada como uma coleção
%de conceitos interligados entre si. Esses conceitos são fundamentados para
%possibilitar a descoberta e reuso do conhecimento sendo escrito.
%
%Existem diferentes maneiras de se escrever essa fundamentação, entretanto a
%adotada no presente trabalho foi baseada em OWL (\emph{Ontology Web
%Language}). Essa linguagem foi regulamentada pela W3C\footnote{Ver
%\url{http://www.w3.org/standards/semanticweb/ontology}.}, orgão internacional
%que regulamenta padrões na Web, para ser usada na Web Semântica. Essa
%linguagem de descrição foi criada em 2002 e trabalha com a possibilidade do
%mundo aberto. Logo, uma coisa só é verdadeira ou falsa quando foi
%explicitamente dito isto, caso contrario não se sabe nada.
%
%Assim, utilizando essa ferramenta será construída uma ontologia do modelo
%afetivo definido por \citet{ortony1988cse}. Esse modelo fundamenta ao todo 22
%emoções diferentes divididos em três formas de percepção: consequência,
%responsabilidade e atratividade. Além disso, o presente trabalho pretende
%estudar e relatar outras ontologia afetivas e introduzir a linguagem OWL em
%maior detalhe.

% a parte comentada vai na introducao propriamente dita...
% isso aqui seria o trailer

\section{Resumo}
Segundo \citet{ontoly2004Approach}, ontologia é a representação de um
conhecimento utilizado para se capturar informações sobre determinado assunto
e que é, normalmente, estruturado de forma similar a uma rede semântica. Dessa
forma, uma ontologia permite a descoberta, interoperabilidade e reuso do
conhecimento sendo escrito.
%
O presente trabalho pretende estudar e relatar outras ontologias afetivas,
utilizar a linguagem OWL\footnote{Ver mais em
\url{http://www.w3.org/standards/semanticweb/ontology}.} para descrever o
modelo afetivo definido por \citet{ortony1988cse} e que fundamenta 22 emoções
divididas em três formas de perceber o mundo.


\section{Cronograma das atividades}

	\begin{tabular}[c]{c|cccccc}
		Atividades & \begin{sideways} \small{Jul/11} \end{sideways}& \begin{sideways} \small{Ag/11} \end{sideways}& \begin{sideways} \small{S/11} \end{sideways}& \begin{sideways} \small{O/11} \end{sideways}& \begin{sideways} \small{N/11} \end{sideways}& \begin{sideways} \small{D/11} \end{sideways} \\ \hline
		%\cellcolor{gray!50} pendente & \\
		%\cellcolor{green!50} feito & \\
		%\cellcolor{blue!50} andamento & \\
		%\cellcolor{red!85} atrasado &  \\
	Estudo bibliográfico & \cellcolor{gray!50} & \cellcolor{gray!50} & \cellcolor{gray!50} & \cellcolor{gray!50} &  \\
	Elaboração proposta &  & \cellcolor{gray!50} &  &  &  &  \\
	Desenvolvimento da ontologia &  &  & \cellcolor{gray!50} & \cellcolor{gray!50} & \cellcolor{gray!50} &  \\
	Redação da monografia &  & \cellcolor{gray!50} & \cellcolor{gray!50} & \cellcolor{gray!50} & \cellcolor{gray!50} & \cellcolor{gray!50}
	\end{tabular}

%\bibliographystyle{abnt-alf}
\bibliography{mestrado}

	\vfill

   \begin{center}
		\rule{8cm}{.1mm} \\ Orientador: Prof.~Dr.~Rafael Heitor Bordini
    \end{center}
   \begin{center}
		\rule{6cm}{.1mm} \\ Aluno: Ricardo Rodrigues Lucca
    \end{center}

\end{document}
