\chapter{Trabalho Proposto} \label{cap:tp}
%2345678901234567890123456789012345678901234567890123456789012345678901234567890

Esse capítulo visa mostrar de maneira organizada a ideia proposta e
desenvolvida. Dessa forma, a primeira seção apresenta a ideia geral do
trabalho sendo proposto. A seguir, a ontologia do modelo afetivo é
apresentada. Em seguida, a ontologia do modelo de percepções é mostrado. Por
fim, a ontologia do modelo de humanos virtuais é discutida e como foi feito a
junção com a mesma. \todo{esse paragrafo precisa ser revisto no final do cap}
% cuidar com as palavras: seção, ontologia, modelo

\section{Ideia Geral}

\todo{o nome da secao parece vago!}Apresentacao geral do que eh feito.
Talvez fazer um ou dois paragrafos e colocar no inicio do capítulo?

\section{Ontologia Afetiva} \label{cap:tp:oa}

...

\section{Ontologias do Ambiente} \label{cap:tp:oda}
% de percepcoes virou do ambiente

...

\section{Reutilizando uma Ontologia de Humanos Virtuais} \label{cap:tp:ruodhv}

...
